\chapter{Quedas}

\section{Introdução}

...contextualizar as quedas...

\section{Resultados}

Também no ano de 2019, 151 moradores de Campinas faleceram em
decorrência de uma queda, o que resultou em um coeficiente de mortalidade padronizado de 7,0 óbitos para cada 100 mil moradores.

Com relação ao sexo das vítimas, 79 (52,3\%) eram homens e 72 (47,7\%) eram mulheres. Essa distribuição resultou em um coeficiente padronizado de mortalidade de 11,7 e 3,2 óbitos para cada 100 mil homens e mulheres, respectivamente. Com relação à cor da pele, dentre os mortos, 104 (68,9\%) eram amarelos ou brancos, enquanto 47 (31,1\%) eram pardos ou pretos.

A Tabela \# mostra estatísticas obtidas ao analisarem-se os dados na forma de um estudo caso-controle, sendo o grupo de casos composto pelos mortos devido a quedas e o grupo controle constituído pela população viva amostrada.

Tabela \#: estatísticas obtidas no estudo caso-controle tendo como casos os falecidos por quedas e como controles a população viva amostrada.

Pode-se interpretar esses resultados da seguinte forma:

\begin{itemize}
    \item Referir medo de sofrer violência por parte de criminosos e/ou policiais no último mês está associado a uma redução de 89\% do risco de morrer em decorrência de queda.
    \item Um aumento de um ano de escolaridade está associado a uma redução de 9\% do risco de morrer em decorrência de queda.
    \item O uso problemático de álcool está associado a um aumento de 21,5 vezes do risco de morrer em decorrência de queda.
    \item Homens têm risco de morrer em decorrência de queda 2,15 vezes maior do que mulheres.
    \item Um aumento de um ano de vida está associado a um aumento de 8\% do risco de morte em decorrência de uma queda.
\end{itemize}

A Figura \# mostra a distribuição do local de ocorrência das quedas que levaram ao óbito em 2019 nas regiões classificadas como pobre, média e rica da cidade. Levando-se em consideração a população dessas regiões, os coeficientes padronizados foram 10,8; 7,7 e 5,4 por cem mil habitantes, respectivamente. Assim, o risco de sofrer uma queda fatal nas regiões pobre e média, em relação à região rica da cidade, foi estimado em 2,0 e 1,4 respectivamente.

A Tabela \# apresenta coeficientes de mortalidade padronizados e estimativas de risco relativo de morte por queda segundo regiões da cidade e sexo.

Tabela \#: Coeficientes de mortalidade padronizados e riscos relativos de homicídios segundo áreas da cidade e sexo.

Obtida de maneira análoga à Figura \#, a Figura \# mostra como se distribui pelo território de Campinas o risco de morrer em decorrência de uma queda, relativo ao risco médio de morte por queda no município. A figura evidencia áreas de proteção a noroeste e no centro da cidade, bem como áreas de risco principalmente ao sudeste e sudoeste.

\section{Discussão}

... discussão geral das quedas...
