\chapter{Feminicídios}

\section{Introdução}

Os homicídios são crimes que afetam as vítimas, os autores, as famílias, a comunidade e a sociedade no conjunto. Constituem um dos indicadores mais utilizados para a contagem e monitoramento de mortes violentas (United Nations Office on Drugs and Crime (UNODC), 2019 Global Study on Homicide, Vienna). Um homicidio é uma morte ilegal infringida sobre uma pessoa com a intensão de causar morte ou ferimento serio. Há três elementos que caracterizam um homicídio intencional: a morte de uma pessoa por outra (elemento objetivo), a intenção da pessoa que matou ou feriu a vítima (elemento subjetivo) e a ilegalidade da morte (elemento legal) (United Nations Office on Drugs and Crime (UNODC), 2019 Global Study on Homicide, Vienna). Os homicídios são a principal causa de mortes em pessoas de 15 a 29 anos ocasionando grandes perdas na expectativa de vida e altos custos sobre os sistemas de saúde. 

As principais vítimas e agressores dos homicidios são notadamente do sexo masculino. No mundo em 2017 estimou-se uma taxa de 6,1 homicídios por cada 100,000 habitantes. As taxas mais altas podem ser vistas nas Américas (17,2/100,000 hab.) e na África (13/100,000 hab.) (United Nations Office on Drugs and Crime (UNODC), 2019 Global Study on Homicide, Vienna). Em relação ao sexo feminino, estimaram-se no mesmo ano 87,000 mortes de mulheres e meninas vítimas de homicídio, das quais 58\% morreram vítimas dos parceiros ou outros membros da família. Os continentes com mais mortes de mulheres são a África (3,1 homicídios/100,000 mulheres) e as Américas (1,6 homicídios/ 100,000 mulheres) (United Nations Office on Drugs and Crime (UNODC), 2019 Global Study on Homicide, Vienna). Sociedades com grande iniquidade entre os  gêneros são caracterizadas por ter altos níveis de violência contra as mulheres.

As mortes vinculadas a inequidade e estereótipos de gênero são nomeadas feminicídios. Um feminicídio se corresponde com o assassinato de uma mulher devido a sua condição feminina, ou gênero feminino. Os feminicídios, são o resultado de múltiplas, crescentes e contínuas manifestações de violência, que estão enraizadas historicamente nas relações desiguais de poder entre homens e mulheres e na discriminação sistêmica do gênero feminino que se sustenta nos valores sociais, religiosos, econômicos e em práticas culturais (Ellsberg M, Peña R, Herrera A, Liljestrand J, Winkvist A. Candies in hell: women’s experiences of violence in Nicaragua. Social science & medicine. 2000;51(11):1595-610). 

As mulheres independentemente da sua condição econômica, idade ou educação estão expostas à violência (United  Nations. The world’s women 2015: trends and statistics. New York (NY): United Nations, Department of Economic and Social Affairs. Statistics Division. 2015.). Mundialmente 30\% das mulheres que estiveram em um relacionamento relataram ter experimentado alguma forma de violência física e/ou sexual cometida pelo parceiro íntimo durante a vida (World Health Organization. Global and regional estimates of violence against women: prevalence and health effects of intimate partner violence and non-partner sexual violence: World Health Organization; 2013.). Assim, o principal fator de risco para homicídio por parceiro íntimo é a violência doméstica prévia (Campbell JC, Glass N, Sharps PW, Laughon K, Bloom T. Intimate partner homicide: Review and implications of research and policy. Trauma, Violence, & Abuse. 2007;8(3):246-69.).

Os feminicídios não constituem eventos isolados, repentinos ou inesperados; ao contrário, fazem parte de um processo contínuo de violências, cujas raízes misóginas caracterizam o uso de violência extrema (Instituto Patricia Galvão. Dossiê Feminicídio. O que é Feminicídio?  [Available from: https://dossies.agenciapatriciagalvao.org.br/feminicidio/capitulos/o-que-e-feminicidio/.)Acontecem principalmente em contextos intrafamiliares e domésticos. Podem ser perpetrados por parceiros ou ex-parceiros íntimos, membros da família e, em raras ocasiões, por mulheres (Weil S, Corradi C, Naudi M. Femicide across Europe: Policy Press; 2018.). O lar é o lugar mais provável para uma mulher se tornar vítima de homicídio, enquanto os homens têm maior probabilidade de ser assassinado na rua ou em lugares públicos (United Nations Office on Drugs and Crime. 2011 Global Study on Homicide: Trends, Contexts, Data: United Nations Office on Drugs and Crime; 2011.). 

Não todos os homicídios femininos são feminicídios, só aqueles nos que é possível identificar uma lógica de relações desiguais de poder entre os gêneros (Carcedo A. No olvidamos ni aceptamos: femicidio en Centroamérica, 2000-2006: Asociación Centro Feminista de Información y Acción (CEFEMINA); 2010.). Feminicídios acontecem em todos os países do mundo. Em 2019 se contabilizaram 137 mulheres mortas diariamente por um membro da própria família (United Nations Office on Drugs and Crime. Global Study on Homicide: Gender-related Killing of Women and Girls Viena: UNODC, United Nations Office on Drugs and Crime; 2019 [Available from: https://www.unodc.org/unodc/en/data-and-analysis/global-study-on-homicide.html.) e suspeita-se que a quantidade de feminicídios estão subestimados, alguns deles encobertos como casos decorrentes de suicídio ou de mortes acidentais. 

No mundo 38\% dos assassinatos de mulheres motivados pelo gênero são cometidos por um parceiro íntimo ou parente de sexo masculino (United Nations Office on Drugs and Crime. 2011 Global Study on Homicide: Trends, Contexts, Data: United Nations Office on Drugs and Crime; 2011.). No Brasil, estimou-se uma proporção de feminicídios de 29,6\% (Fórum Brasileiro de segurança pública. Anuário Brasileiro de segurança pública 2019. 2019.). No atlas da violência de 2019 foi reportado o crescimento dos homicídios femininos; estimaram-se 13 assassinatos de mulheres por dia, sendo que, no período 2007-2017 houve aumento de 30,7\% do número de casos, fechando com uma taxa nacional de 4,7 assassinatos por cada 100,000 mulheres. 

A raça também é um elemento importante na análise dos feminicídios. Mulheres negras são mais atingidas. Enquanto a taxa de homicídios de mulheres não negras teve crescimento de 1,6\%, a taxa de homicídios das mulheres negras cresceu 29,9\% entre 2007-2017 (Instituto de Pesquisa Econômica Aplicada, Fórum Brasileiro de Segurança Pública. Atlas da Violência. Brasília; 2019.).

Mulheres vítimas de violência, em grande medida, estão envolvidas emocionalmente ou são economicamente dependentes de quem as vitimiza, contribuindo para a perpetuação e aceitação da violência (Ellsberg M, Peña R, Herrera A, Liljestrand J, Winkvist A. Candies in hell: women’s experiences of violence in Nicaragua. Social science & medicine. 2000;51(11):1595-610., Krug EG, World Health Organization. World report on violence and health. Geneva2002.). Um número reduzido de mulheres que sofre violência busca ajuda; principalmente elas recorrem a amigos, parentes ou instituições de saúde. A proporção de mulheres que procura por ajuda da polícia é inferior a 10\%, entre os motivos para a não procura estão: a baixa proporção de policiais do sexo feminino, falta de acessibilidade dos serviços, medo ou vergonha, acreditar que a polícia não pode fazer nada a respeito e o desejo de manter o incidente violento no privado (United  Nations. The world’s women 2015: trends and statistics. New York (NY): United Nations, Department of Economic and Social Affairs. Statistics Division. 2015.). Assim como as dificuldades de acesso à justiça e o baixo número de punições efetivas dos agressores.

\section{Resultados}

Em 2019 aconteceram 18 casos de homicídio feminino, dentro deles 13 (72\%) puderam ser caracterizados como feminicídio. Houve a morte de duas mulheres transgénero. Na tabela X pode ser vista uma descrição das caraterísticas sociodemográficas das mulheres, os mecanismos de morte e principais motivações dos assassinatos.
Tabela X Descrição dos casos de feminicídio e homicídio na cidade de Campinas 2019
Casos de feminicídio
Caso 1 
Mulher de 75 anos, parda, sem estudos, casada e atendente de bar em bairro
Provável motivador da morte: Desentendimento com filho Meio de morte: Espancamento e empurrão com TCE
Caso 2
Adolescente de 13 anos, parda, com ensino fundamental incompleto, em união estável e atendente de serviços
Provável motivador da morte: Desentendimento do casal, ele não queria que ela saísse de casa Meio de morte: Impacto de bala na coxa e grande perda sanguínea 
Caso 3 
Mulher de 54 anos, branca, com ensino médio incompleto, solteira e comerciante 
Provável motivador da morte: Término de relacionamento Meio de morte: Queimadura com gasolina do 95% superfície corporal
Caso 4 
Mulher de 40 anos, negra, casada/união estável e catadora de reciclagem
Provável motivador da morte: Desentendimento do casal Meio de morte: Espancamento e queimadura com gasolina do 80% superfície corporal
Caso 5 
Senhora de 82 anos, branca, com ensino fundamental incompleto, viúva e aposentada
Provável motivador da morte: Sofreu agressões do ex-namorado na neta por término de relacionamento Meio de morte: Golpes múltiplos com várias armas brancas
Caso 6 
Adolescente de 19 anos, parda, com ensino fundamental incompleto, solteira, vinculada ao tráfico
Provável motivador da morte: Vínculo e dívidas com o tráfico Meio de morte: 31 perfurações no tórax e abdômen com sinais de agressão sexual
Caso 7 
Mulher de 21 anos, negra, solteira e tesoureira de um supermercado
Provável motivador da morte: Término de relacionamento Meio de morte: 15 impactos de bala
Caso 8 
Mulher de 28 anos, parda, com ensino superior incompleto, separada e atendente de serviços
Provável motivador da morte: Termino de relacionamento Meio de morte: Enforcamento, simulação de suicídio
Caso 9 
Mulher de 38 anos, branca com estudos de pós-graduação, solteira.
Provável motivador da morte: Ciúmes por suspeita de ela ter outro relacionamento Meio de morte: 16 impactos de bala
Caso 10 
Mulher de 31 anos, branca, com ensino fundamental incompleto, solteira e atendente de serviços
Provável motivador da morte: Término de relacionamento e suspeita de ter relacionamentos com outros homens Meio de morte: Espancamento e agressão com faca na região cervical
Caso 11 
Mulher de 30 anos, parda, com ensino superior incompleto, solteira e atendente de serviços
Provável motivador da morte: Desentendimento por dinheiro com companheiro de 4 meses de relacionamento Meio de morte: Ferimentos cervicais, torácicos e abdominais por arma branca
Caso 12
Mulher trans de 35 anos, parda, atendente de bar
Provável motivador da morte: Agressor com problemas de saúde mental quem argumentou que ouvia vozes que lhe indicaram matar a mulher. Meio de morte: Traumatismo cranioencefálico, com afundamento do crânio e ferimento no peito onde foi retirado uma parte do pulmão
Caso 13
Mulher trans de 36 anos, branca, solteira e trabalhadora sexual
Provável motivador da morte: Desentendimento com cliente Meio de morte: Espancamento e trauma contundente na cabeça com garrafa
Casos de homicídio de mulher
Caso 14
Mulher de 28 anos, parda, com ensino fundamental incompleto, solteira e faxineira
Provável motivador da morte: Desconhecido entrou em casa procurando outro homem, ela fez resistência Meio de morte: Um impacto na boca com arma de fogo
Caso 15
Mulher de 43 anos, com ensino médio incompleto, solteira e vinculada ao tráfico de drogas
Provável motivador da morte: Acerto de contas por tráfico Meio de morte: 16 impactos com arma de fogo semiautomática
Caso 16
Mulher de 27 anos, branca, casada e vinculada ao tráfico de drogas
Provável motivador da morte: Acerto de contas por tráfico Meio de morte: 3 golpes de faca no pescoço
Caso 17
Mulher de 50 anos, branca, casada e empregada doméstica
Provável motivador da morte: latrocínio Meio de morte: 16 golpes de faca com perfuração do fígado
Caso 18
Senhora de 86 anos, branca, casada e aposentada
Provável motivador da morte: latrocínio Meio de morte: Múltiplos ferimentos de faca, com um golpe fatal na região torácica esquerda

A mediana de idade das mulheres foi de 35,5 anos sendo a mulher mais jovem de 13 anos e a de maior idade de 86 anos. Delas 50\% (n=9) de raça branca, 39\% (n=7) de raça parda e 11\% (n=2) de raça preta.  As mulheres professavam crenças católicas em 55\% dos casos e evangélicas em 33\%. 56\% das mulheres eram de fora do estado de São Paulo e 12 tinham filhos.
Das 18 mulheres, 13 (72\%) desempenhava algum tipo de atividade remunerada, mas só uma delas tinha um vínculo de trabalho formal com carteira assinada e direitos trabalhistas e uma das duas mulheres idosas era aposentada. O local de morte mais relevante foi o domicílio da vítima 66\% (n=12), as outras mortes aconteceram em via pública e comércio 33,3\% (n=6). As mulheres foram mortas utilizando vários mecanismos juntos, mas notadamente por arma branca em 44\% (n=8) dos casos, seguido por arma de fogo 28\% (n=5) e 11\% (n=2) por queimadura. Outros mecanismos como enforcamentos, espancamentos e golpes com objetos contundentes apareceram no restante dos casos. 
Nos casos 1-13 correspondem a feminicídios, em 61\% (n=8) deles o agressor era parceiro ou ex-parceiro da mulher, em um caso o filho e em 2 casos um cliente. Só em 2 casos houve algum tipo de denúncia às autoridades por agressões prévias das quais só uma obteve a medida protetiva, mas 61\% (n=8) das mulheres sofreram algum tipo de violência doméstica. 
Entre os 5 casos não considerados feminicídios dois foram vinculados ao tráfico, dois foram caracterizados como latrocínios e no último o agressor matou a mulher por fazer resistência ao tentar proteger o filho. 


\section{Discussão}

No Brasil a mortalidade por homicídios de mulheres integram cerca de 10\% do total dessas mortes, fato que por muitos anos silenciou estudos sobre o tema. Além da violência contra a mulher, ser naturalizada e tolerada, tornando-se banalizada e socialmente aceitável (da Violência, I. A. (2018). Disponível em: http://www. ipea. gov. br/atlasviolencia/link/7/crimes-violentos-contra-a-pessoa.). Porém, a menor magnitude dos homicídios de mulheres não confere aos crimes, importância secundária.  Pelo contrário, a violência contra à mulher é um problema de saúde pública, reconhecido pela Organização Mundial da Saúde como tal, desde 1990 (OMS. Organização Mundial da Saúde. Folha informativa – violência contra as mulheres. 2016.) Cada vez mais este fenômeno exige atenção por parte das autoridades, estudos acadêmicos e políticas públicas direcionadas  (Meneghel & Hirakat. “Femicides: female homicide in Brazil”, 2011).

O coeficiente de mortalidade padronizado para homicídio feminino (tabela #) mostra um efeito de desigualdade econômica e social. Entre mulheres pobres o coeficiente é de 5,4 comparada com 2,3 nas mulheres da classe econômica mais alta o que representa um risco de morte 140\% neste grupo.    
As vítimas foram mulheres jovens (mediana 35,5 anos) economicamente ativas, estudos têm mostrado que o homicídio é uma das principais causas de anos potenciais de vida perdidos em mulheres de 10 a 39 anos, sendo mais relevantes do que as mortes por doenças cardiovasculares e neoplasias (Arnold et al, 2007; Amaral et al, 2011). Para o Brasil, observa-se uma baixa incidência de homicídios femininos até os 10 anos de idade, um crescimento importante dos 12 aos 30 anos  e, em seguida um declínio até a velhice (Mapa da Violência, 2015).

\subsection{Feminicídio}

Um estudo prévio (prévio ou anterior?) do epiGeo mostrou que no ano de 2015 em Campinas se registraram 185 homicídios, sendo 26 de mulheres (14,1\%)  e dentre eles, 19 feminicídios (73,1\%) o que correspondeu a um coeficiente de mortalidade de 3,2 por 100,000 mulheres (Caicedo-Roa, Monica, Cordeiro, Ricardo Carlos, Martins, Ana Cláudia Alves, & Faria, Pedro Henrique de. (2019). Femicídios na cidade de Campinas, São Paulo, Brasil. Cadernos de Saúde Pública, 35(6), e00110718. Epub July 04, 2019.https://doi.org/10.1590/0102-311x00110718). No presente estudo, do total de 153 homicídios, 18 foram homicídios de mulheres; dos quais 13 foram classificados como feminicídio (72,0\%), ou seja, assassinatos em que houve um fator gênero relacionado ao crime. Resultando em um coeficiente de mortalidade por  feminicídio de X por 100,000 mulheres em 2019, correspondendo à XXX mortes em cada 100,000 mulheres no ano. 

Mulheres, vítimas de feminicídio também são maioritariamente jovens. 61\% negras (pretas e pardas) e 39\% de pele branca; solteiras, com filhos.  Esses achados para Campinas, acompanham os dados de feminicídio do Brasil, onde as mulheres negras aparecem como a maioria  das vítimas. Segundo o Mapa da Violência 2015, o número de mulheres negras vítimas de morte violenta aumentou 54\% em dez anos, passando de 1,864, em 2003, para 2,875, em 2013. Além da violência doméstica, o racismo aparece como um fator importante, evidenciando as desigualdades estruturais existentes no país, o preconceito histórico contra a população negra gera condições de vida muito desiguais, além de aumentar o risco de violência fatal. Segundo Passinato (2011), ao se estudar os feminicídios, além das discriminações baseadas em gênero, é necessário abranger as intersecções entre gênero, classe social, geração, deficiências, raça, cor e etnia (PASINATO, Wânia. “Femicídios e as mortes de mulheres no Brasil”. In Cadernos Pagu, nº 37, 2011, pp. 219-246. Disponível em http://www.scielo.br/pdf/cpa/n37/a08n37.pdf).

Segundo Diretrizes Nacionais para investigação dos feminicídios, entre as múltiplas razões de desigualdade de gênero na prática das mortes violentas das mulheres estão: o sentimento de posse do agressor sobre a mulher, o controle do corpo, o tratamento da mulher como objeto sexual, o desprezo pelo feminino, assim como a tentativa do agressor em eliminar a mulher de forma profissional, econômica, social e intelectual (Diretrizes para Investigar, Processar e Julgar com Perspectiva de Gênero as Mortes Violentas de Mulheres – Feminicídios - ONU Mulheres, 2016). O que podemos relacionar com a maioria de mulheres economicamente ativas sendo as principais vítimas. No grupo estudado só uma mulher era aposentada, as outras desempenhavam trabalhos remunerados, mesmo que algums deles não reconecidos, como atividades vinculadas ao tráfico e trabalho sexual; só uma das mulheres tinha vinculo de emprego formal. Antigamente a mulher era ainda mais dependente social e economicamente ao homem. Nos dia de hoje, as mulheres têm ganhado espaço no mercado de trabalho, mesmo com as problemáticas que ainda precisam ser superadas, como a falta de ascensão aos cargo diretivos e de gestão e a iniquidade salarial. A saída da mulher do ambiente doméstico e a crescente independência financeira gera conflitos ao sistema patriarcal sendo um agravante para a aparição de casos de violência doméstica e feminicídio.

Um dos papéis sociais altamente valorados da mulher e sua função como mãe e cuidadora. A maioria das vítimas de feminicídio no estudo tinha um ou mais filhos. O feminicídio além de tirar brutalmente a vida da mulher, atinge outras pessoas da família ou terceiros que são consideradas vítimas secundárias. Filhos das vitimas ficam sem sua cuidadora principal, tanto eles quanto as pessoas próximas muitas vezes presenciam o crime, o que pode ter consequências psicológicas, como doenças mentais, uso problemático de drogas, auto agressão e suicídio. Ou também, acabam sendo vítimas fatais do agressor (Raio x do feminicídio, 2018; Lysell et al., 2016).

Os feminicídios acontecem tanto no âmbito privado como no público, porém o lar configurou o espaço mais perigoso para as mulheres, sendo que 66\% das mulheres foram mortas no domicílio. A vitimização das mulheres dentro de casa, longe dos olhos da sociedade, reforça o caráter privado dos crimes, ao mesmo tempo que os meios empregados sugerem a desvantagem física e a desproteção da vítima em relação ao agressor (Pasinato W. Diretrizes Nacionais Feminicídio: investigar, processar e julgar com perspectiva de gênero as mortes violentas de mulheres. 2016.). Segundo o Atlas da Violência (2020), entre 2017 e 2018, o percentual de mulheres mortas dentro da residência foi 2,7 maior do que de homens (Instituto de Pesquisa Econômica Aplicada, Fórum Brasileiro de Segurança Pública. Atlas da Violência. Brasília; 2020).

No raio X do feminicídio o núcleo de gênero do Ministério público do Estado de São Paulo levantou 364 casos de mortes violentas de mulheres entre março de 2016 e março de 2017. Do total de casos estudados 66\% foram tentativas e 34\% crimes consumados de feminicídio. O meio maiormente empregado para os ataques foram armas brancas (facas, foices e canivetes) em 58\% dos casos e em 17\% foram utilizadas armas de fogo ( (Raio x do feminicídio, 2018). Nos achados da pesquisa, na maioria das mortes por feminicídio os crimes foram cometidos com arma branca (38,4\%) e de fogo (23\%).  A disponibilidade de armas de fogo é um fator que propicia agressões potencialmente fatais.

Os perpetradores dos crimes foram homens ligados às vítimas, destacando os atuais companheiros (38,5\%) ou ex-companheiros (23\%), outros parentes (15,4\%) e vinculados à atividade econômica (23\%). Em concordância com os achados no Brasil, a maior incidência de feminicídio ocorreu entre pessoas que têm, ou tiveram, uma relação de união estável (70\%), seguido por namorados ou ex-namorados (12\%). As principais motivações dos crimes foram separação ou rompimento do relacionamento (45\%), atos de ciúmes/machismo (30\%) e discussões banais (17\%) (Núcleo de Gênero Ministerio Público do Estado de São Paulo. Raio X do feminicídio em São Paulo  É possível evitar a morte   2018 [Available from: http://www.compromissoeatitude.org.br/raio-x-do-feminicidio-em-sao-paulo-promotora-valeria-scarance-reforca-que-e-possivel-evitar-morte/). 

Destacam-se duas mortes de mulheres trans, vítimas do preconceito por sua identidade transexual. De raça branca e parda, com 35 e 36 anos de idade, mortas por agressões com  garrafa de vidro e espancamentos, as duas com requintes de crueldade. Mortes de mulheres trans  são silenciadas na estimativa oficial do número de casos, pois muitas vezes essas informações sobre a identidade de gênero não são especificadas. Segundo a ONG Transgender Europe (TGEU), a partir de materiais levantados pela imprensa, o Brasil foi colocado como o país que mais mata a população trans e travesti (ONG Transgender Europe (TGEU). Entre 2008 e 2016 foram 845 mortes, 42\% de todo o mundo. Segundo o Dossiê: Assassinatos e violência contra travestis e transexuais brasileiras em 2019, o Brasil levantou 124 assassinatos de pessoas trans, sendo 121 travestis e mulheres transexuais, uma taxa de 4,8 assassinatos para cada 100,000 habitantes, 6 vezes maior do que a taxa para os Estados Unidos. São Paulo, apareceu como o Estado que mais matou, 21 casos em 2019 (Dossiê dos ASSASSINATOS e da violência contra TRAVESTIS e TRANSEXUAIS no Brasil em 2019.). A violência chama atenção em todas as idades, porém em maior magnitude entre os 15 e 45 anos, as travestis e transexuais femininas são um grupo de alta vulnerabilidade à morte violenta. Segundo Antunes (2013), a expectativa de vida desta população é de 35 anos, enquanto a população brasileira em geral, é de 74,9 anos, 40 anos a menos (Antunes, Pedro Paulo Sammarco. 2013. Travestis envelhecem? São Paulo: Annablume; IBGE, 2013). Existe a naturalização da violência e a possibilidade da violência dentro ou fora de casa, devido à conjugação de vários preconceitos. As profissionais do sexo, aparecem como maioria, 67\% das vítimas. Além de sofrem a marginalização do Estado, elas são mais expostas à violência direta e a somatória de estigmas com a profissão. Nos dados coletados as duas mulheres trans (casos 12-13) tinham desenvolvido trabalho sexual.

Em 2006 foi  sancionada a lei 11.340 que cria mecanismos para coibir a violência doméstica e familiar contra a mulher (Lei Nº 11.340, de 7 de agosto de 2006, (2006) comumente conhecida como lei Maria da Penha.  Nela são reconhecidos os direitos fundamentais inerentes à pessoa humana das mulheres, independentemente de classe, raça, etnia, orientação sexual, renda, cultura, nível educacional, idade e religião. Reconhece todas as formas de violência doméstica contra a mulher e determina as disposições sobre a assistência à mulher em situação de violência doméstica e familiar, incluindo as medidas protetivas, os benefícios de proteção, acompanhamento jurídico, proteção em casas abrigo, centros de educação e reabilitação dos agressores e ações de educação e de estabelecimento de estatísticas e pesquisas sobre violência contra a mulher (Lei Nº 11.340, de 7 de agosto de 2006, (2006).

Posteriormente foi sancionada a Lei Nº 13.104, de 9 de março de 2015 (lei de feminicídio). Lei que altera o código penal, para prever o feminicídio como circunstância qualificadora do crime de homicídio, e inclui o feminicídio no rol dos crimes hediondos (Lei nº 13.104, de 9 de Março de 2015, (2015) com uma pena de reclusão de 12 a 30 anos. O crime é tipificado como feminicídio si é praticado contra a mulher por razoes da condição de sexo feminino (violência doméstica e familiar e/ou menosprezo ou discriminação à condição de mulher) e aumenta a pena em um terço e até a metade se o crime for praticado durante a gestação ou nos três meses posteriores ao parto; contra pessoa menor de 14 anos, maior de 60 anos ou com deficiência ou na presença de descendente ou de ascendente da vítima. 

Nos dados levantados, somente duas vítimas haviam feito algum tipo de denúncia às autoridades por agressões prévias, o que corresponde a 15,4\% dos casos de feminicídios. No Brasil, em 2018 estimou-se que só o 4\% das mulheres mortas por causa violenta tinham registrado um boletim de ocorrência contra o agressor, delas só 3\% tinha medida protetiva(Núcleo de Gênero Ministério Público do Estado de São Paulo. Raio X do feminicídio em São Paulo  É possível evitar a morte   2018 [Available from: http://www.compromissoeatitude.org.br/raio-x-do-feminicidio-em-sao-paulo-promotora-valeria-scarance-reforca-que-e-possivel-evitar-morte/.)o que mostra as dificuldades das mulheres para denunciar e consequentemente o limitado alcance da lei Maria da Penha. Além de que mais da metade das mulheres mortas, já haviam sofrido violência doméstica (61\%), dados que reforçam o feminicídio como morte evitável, por ser uma violência fatal, previamente anunciada. Destaca-se a importância da denúncia para quebrar o ciclo de violência (Kelly, 1988; Meneghel, 2017) e exigir do Estado a atuação ativa na aplicação da lei e das estratégias para a prevenção da morte das mulheres.

\subsection{Homicídios femininos vinculados ao tráfico} 

Os crimes relacionados ao mundo do crime representam 75-90\% das mortes violentas no Brasil e as vítimas são majoritariamente homens jovens e negros (Ref. Texto do Feltran sobre as tipificações). Durante o período de 2005 a 2016, observou um aumento de 227,7\% do encarceramento de mulheres, sendo que 62\% das mulheres presas estão condenadas ou aguardam julgamento por acusações relacionadas ao tráfico de drogas (Ministério da justiça, 2017). Em Campinas houve dois casos de vítimas mortas em decorrência de acerto de contas pelo tráfico. As vítimas eram mulheres adultas, uma morta com perfurações por arma branca e a outra por arma de fogo (16 disparos). Mesmo que o tráfico tenha mais vítimas de sexo masculino, as mulheres também são atingidas por esse tipo de violencia. Não foram achados estudos específicos de mortes femininas vinculadas ao tráfico, mas cada vez há mais participação feminina neste tipo de atividade (tese: Melo 2018. A dona da boca, a vendedora e a mula).

\subsection{Homicídios femininos em decorrência de latrocínio}

Dois casos foram classificados como latrocínios, o roubo seguido de morte. As vítimas eram mulheres de 50 e 86 anos, brancas, casadas que morreram em suas residências em decorrência de múltiplos golpes de faca durante o roubo, essas mortes costumam representar em torno de 3,3\% das mortes violentas no Brasil  (Ref. Texto do Feltran sobre as tipificações).

\section{Considerações finais}

Esse capítulo teve como intenção discutir os aspectos mais relevantes das mortes por homicídios femininos e feminicídios em 2019. Com a ajuda da autópsia verbal foi possível aprimorar as informações coletadas pela secretaria de saúde e contextualizar os crimes. Reforça-se que essas mortes são evitáveis e de responsabilidade do Estado. Ao fazermos a diferenciação das mortes de mulheres, dentro de todos casos de  homicídio, além de conhecermos a magnitude real do problema da violência contra a mulher na cidade, conseguimos analisá-la como fenômeno complexo, multifatorial e enraizado nas relações e nos ambientes privados e públicos. 

Tanto as receitas para morrer por amor quanto o conto o coração, mil reais e um rádio de pilha no começo do capitulo fizeram uso de um recurso literário para ilustrar as realidades das mulheres em Campinas. O proposito mais do que ressaltar a violência, os mecanismo de morte e suas motivações banais foi problematizar e discutir a complexidade da violência contra as mulheres que termina em mortes prematuras e injustas. Discutir questões de gênero é fundamental para o  entendimento e a prevenção de casos que se entrelaçam com outras problemáticas próprias das relações e da dinâmica da cidade. 
