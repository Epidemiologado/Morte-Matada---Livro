\chapter{Recorte Territorial}

Nos 20 primeiros anos do presente século, 13.378 moradores de Campinas morreram em consequência de uma “causa externa”. Essa é uma expressão local da violência social do Brasil. A ocorrência dessas mortes, pormenorizadamente as ocorridas em 2019, será descrita nesse estudo por meio de alguns indicadores de mortalidade. Entretanto, indicadores de saúde vistos em escala municipal são médias velando a paisagem. Saúde e doença não se distribuem homogeneamente dentro das cidades, sobretudo nas grandes. Assim como também não se distribuem de modo homogêneo a infraestrutura urbana, a vigilância policial, o sistema estatal de justiça, o transporte, os espaços de convívio e lazer, as escolas e os serviços de saúde. A dinâmica espacial, com suas variadas formas de segregação urbana, é um componente que descreve mais que só aspetos geográficos, sendo produto e produtor no processo saúde/doença \citep{adorno1998violencias}. Determinantes da iniquidade em saúde tecem espaços urbanos, onde se desenvolvem práticas sociais, econômicas, culturais. Indicadores de mortalidade dizem pouco se não discriminarem, nos diferentes espaços intra-urbanos, quem eram os mortos, onde e como viviam, de que modo e por que, de fato, morreram.

A cidade de Campinas foi decomposta e depois reagrupada segundo o Índice de Desenvolvimento Humano Municipal de suas Unidades de Desenvolvimento Humano. Estas são microrregiões sócio-ambientais relativamente homogêneas, contíguas, contendo ao menos 400 domicílios, cuja identidade é reconhecida pela população nela residente(46).

As Unidades de Desenvolvimento Humano são retratos resumidos de espaços intra-municipais brasileiros, contendo seus principais indicadores socioeconômicos nas áreas de demografia, saúde, educação, habitação, renda, trabalho e vulnerabilidade social, além do Índice de Desenvolvimento Humano Municipal \citep{dos2016metodologia}. Elas foram estabelecidas em todas as regiões metropolitanas do país por técnicos do Instituto de Pesquisa Econômica Aplicada (IPEA), em torno de 2013, como estratégia para melhor entender diferenciais sócio-demográficos intra-urbanos \citep{costa2015atlas}.

O conceito de desenvolvimento humano, bem como sua medida, o Índice de Desenvolvimento Humano, foi apresentado pela primeira vez em 1990 no Programa das Nações Unidas para o Desenvolvimento \citep{UNO2020}, como alternativa ao Produto Interno Bruto Per Capita como medida do grau de desenvolvimento de um país \citep{pnud2015atlas}. Este indicador sintetiza três dimensões populacionais: longevidade, educação e renda. Em 2012, a partir dos dados censitários de 1991, 2000 e 2010, foram calculados IDH para todos os 5.565 municípios brasileiros, sendo esse indicador chamado Índice de Desenvolvimento Humano Municipal (IDHM).

O IDHM é um número real, variando entre zero e um. Quanto mais próximo de um, supostamente maior é o desenvolvimento humano da região a que se refere. Posteriormente, o IDHM foi também calculado para todas as UDH brasileiras(49).

Em Campinas foram definidas 187 UDH em 2013. Calculado a partir dos dados censitários de 2010, o valor do IDHM dessas UDH variou entre 0,636 e 0,954. A Figura \# mostra a distribuição dessas UDH.

Neste estudo, definiram-se três seguimentos de amplitude semelhante de valores de IDHM. O município de Campinas foi dividido em três regiões, contendo as UDH cujos valores de IDHM situam-se no primeiro, segundo e terceiro segmentos dessa distribuição, conforme abaixo especificado:

\begin{itemize}
    \item Primeiro Segmento: 70 UDH S com IDHM variando entre 0,636 e 0,740
    \item Segundo Segmento: 63 UDH S com IDHM variando entre 0,746 e 0,845
    \item Terceiro Segmento: 54 UDH S com IDHM variando entre 0,855 e 0,954
\end{itemize}

Os três segmentos acima definidos poderiam ser chamados de regiões de “IDHM inferior”, “IDHM intermediário” e “IDHM superior”. Entretanto, para simplificar e facilitar as descrições que seguirão, esses segmentos serão doravante chamados simplesmente de regiões “pobre”, “média” e “rica” da cidade, respectivamente. A Figura \# ilustra essa divisão.

Este indicador sintetiza três dimensões populacionais: longevidade, educação e renda. Em 2012, a partir dos dados censitários de 1991, 2000 e 2010, foram calculados IDH para todos os 5.565 municípios brasileiros, sendo esse indicador chamado Índice de Desenvolvimento Humano Municipal (IDHM). [esqueci de falar sobre renda, longevidade e escolaridade como parte do IDHM]
