\chapter{Necropolítica}

... discussão final do livro...

%\section{Necropolítica e papel da família na ocorrência dos eventos.}

Ponto importante para ser levantado: Com o avanço do liberalismo como projeto, a perca do papel do Estado em prestar serviços públicos (como saúde e educação), e a precarização da exploração da força do trabalho (que faz com que cada vez cada um seja empreendedor de si), a segurança pública, os cuidados de saúde mental, a segurança privada, passam a ser cada vez mais centradas nas organizações familiares. Temos uma informação importante nas entrevistas: um desenho (precário mas suficiente) da configuração familiar das vítimas, por meio da autópsia verbal.

Assim pode-se responder à pergunta, que ajuda a definir o papel do Estado na dinâmica de violência: a necropolítica se dá em que momentos por ação ou por omissão do Estado? Se o Estado é omisso, por uma questão de sobrevivência, a família acaba ocupando seu espaço.

É possível identificar relações familiares entre as vítimas e as organizações criminosas responsáveis por sua morte? (quem investiga família e PCC?-Será que nomes como "Família do Norte", "Irmão", são apenas conceitos emprestados da organização familiar clássica ou guardam com ela alguma relação?

Para pensar no impacto governamental sobre a dinâmica das mortes: e quando uma familicia toma o poder? E a irmandade que existe entre os policiais?