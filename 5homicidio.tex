\chapter{Homicídios}

\section{Introdução}


O homicídio é caracterizado pela conduta intencional de uma pessoa que, por uma ação ou uma omissão, venha causar a morte de outra pessoa. Esta conduta está presente em nossa realidade desde tempos imemoriais e pode ser observada nas mais diversas culturas.

% Conceito de Direito Natural e homicídio como resultado dessa repulsa social ainda que sem normativa expressa a respeito. (Lon Fuller - O caso dos exploradores de caverna.) 

%Reale - lições preliminares do Direito - Direito Natural (pg 312)

% Juspositivismo e o homicídio

% Antropologia da violência (clastres)

%Topoligia da violência (Biul chiu Hang)


\sectio{Intencionalidade}

Quando pensamos intencionalidade segundo a regra classificatória prevista no CID-10, no que tange às condutas que possam levar à lesões ou mortes por causas externas, temos que considerá-las intencionais ou não intencionais. As regras colocadas no volume 2º do CID-10 não trazem uma maior especificidade acerca do tema. Por exemplo, quando se fala de intencionalidade da conduta que gera a lesão está se falando de intencionalidade em realizar a conduta (ação ou omissão)? Ou está se falando de intencionalidade de se alcançar o resultado (lesão ou morte)?

Tais inquietações se mostram relevantes quando estudamos a temática da intencionalidade sob a epistemologia jurídica. Quando pensamos a intencionalidade no campo da aplicação de normas jurídicas, a diferenciação entre a intencionalidade quanto à conduta e a intencionalidade quanto ao resultado, nos casos concretos podemos ter diferentes modalidades de tipificação penal.

No que tange à vontade do agressor de realizar a conduta, o verdadeiro exercício da onisciência, que em verdade só existe no campo da religiosidade, é preciso aferir inicialmente se o agressor efetivamente queria realizar a conduta. Nesse primeiro momento, sem pensarmos se queria, ou não, o resultado. O protocolo existente para classificação epidemiológica previsto no CID vai somente até aqui.

Ainda, há que se pensar as situações onde não se tem, por parte da pessoa que realizou a conduta, a intenção. De início, aqui, nem mesmo utilizamos a terminologia "agressor", partindo de um raciocínio de que agressão está vinculada à vontade em realizar a conduta. Pois bem, a discussão merece, nessa abordagem sobre "não intencionalidade", aprofundamento. Aqui, as ciências do Direito merecem ser utilizadas como esteio para a discussão.

1) Dolo eventual
2) Culpa
2.1) negligência
2.2) imprudência
2.3) imperícia
3)Conduta desprovida de dolo ou culpa

Tópico: Classificação de intencionalidade considerando a vontade pelo resultado

1) Crimes materiais, formais e de resultado
2) Dolo específico e dolo simples
3) Dolo eventual
4) Crimes preterdolosos



No Direito Brasileiro o homicídio compõe o grupo os crimes dolosos contra a vida, de competência do Juri Popular. São eles o homicídio, o auxílio, instigação ou induzimento ao suicídio,   o infanticídio, o aborto provocado pela gestante ou com seu consentimento. Destes crimes, apenas o homicídio e o infanticídio são classificados como agressões homicidas pelas classificações médicas. Isso ocorre porque o dolo, isto é, o grau de intencionalidade do agente que causa a agressão apenas é objeto de interesse para o direito. Desta maneira, o suicídio acaba sendo caracterizado na saúde como a lesão autoprovocada, independentemente de se saber se há ou não algum tipo de coação moral ou psicológica de terceiro sobre o morto.


Por outro lado, algumas agressões violentas intencionais também não são registrados como homicídios propriamente ditos. Trata-se dos homicídios ocorridos em transito. Isto pois, novamente, não é possível distinguir, a partir da análise da \textit{causa mortis} se o motorista que abalroou a vítima o teria feito intencionalmente ou não.

Isso corrobora com a análise dos ditos homicídios culposos, aqueles em que não há intenção de matar ou fazer morrer, que geram repercussões jurídicas para os agentes da conduta, podem ou não serem considerados homicídios na visão da saúde. Por exemplo, um amigo que acidentalmente efetua um disparo de arma de fogo contra outro, causará um homicídio aos olhos da saúde e do direito, mas alguém que atropela uma pessoa ao desviar de outro acidente irá compor as estatísticas de acidente de trânsito, ainda que, em eventual processo judicial, tenha de responder por homicídio culposo nos termos do Código de Trânsito Brasileiro. 

Estas distinções entre os homicídios observados a partir da lente do direito ou da saúde são importantes para que se estabeleça de que objeto estamos tratando, e quais suas repercussões para a vida. De maneira geral, tendo por objetivo inicial a obtenção de informações que esclareçam a causa da morte e não a intencionalidade ou motivação para tal evento, a visão sanitária é preponderante neste livro.

%Perguntar se haverá uma parte geral que exporá as causas externas, ou uma tabela que esmiuçe essa questão.

\subsection{Como ocorrem os homicídios?}

Os meios pelos quais pode-se alcançar a morte de outro ser humano são tão extensos quanto a história e a imaginação humana podem fornecer. Com o advento da modernidade e o avanço tecnológico, esses meios tem se tornado cada vez mais eficazes. As armas de fogo, atualmente, tem sido o principal meio empregado nos homicídios - mesmo considerando apenas os ocorridos fora de conflitos armados em razão de guerras.

Ainda que as armas de fogo sejam consideradas formas mais efetivas de levar alguém à morte, seu uso não tornou a prática do homicídio algo mais "civilizado" (Foucault). De fato, %(Referenciar o uso de armas de fogo no como meio cruel)

Por outro lado, pode-se falar em uma verdadeira normalização do fenômeno, no que tange à certas populações específicas. Neste sentido, os homens pretos, pobres, residentes nas periferias dos grandes centros metropolitanos são as principais vítimas dos homicídios. As razões para a preponderância desta população passam por desigualdade social, racismo estrutural, limitação do poder do Estado em certos territórios, abuso de autoridade, entre outras explicações.


O homicídio é, em muitos países, a principal causa de mortalidade dentro dos grupos das causas externas. Isso ocorre na América Latina, e o país não escapa à essa regra. Em 2018, segundo dados do DATASUS, os homicídios correspondiam a X\% das mortes causas externas, totalizando um total de X casos. 

Além disso homicídios variam em razão da realidade social (citar Minayo e Sousa)

A questão das drogas e as disputas em torno da criminalidade também influenciam o evento.

Quando observamos a variação destes eventos no tempo, temos que, embora os noticiários e o imaginário popular normalmente repercutam a importância da segurança em nosso cotidiano, a violência tem historicamente se reduzido. Quando olhamos para a participação da violência sobre o conjunto total de mortes ocorridas temos que existe uma tendência da humanidade reduzir a violência letal (Harari). Em outra vertente, ao olharmos o aspecto mais regional da violência no século XXI, podemos observar uma queda dos homicídios no Brasil a partir de 2004, marcadamente acentuado no Estado de São Paulo.

Esta informações iniciais visam apenas ilustrar que o homicídio, enquanto um evento de magnitude global, nacional e regional, que atinge de forma desigual populações com diferentes características, não pode ser visto, do ponto de vista epidemiológico, como uma doença baseada em um patógeno, ou mesmo um mal crônico, ainda que multicausal. Ele é, assim como outras formas de violência, um fenômeno complexo e dinâmico, que se apresenta em múltiplas facetas a depender do pontos de vista adotado. Destarte, devemos em uma pesquisa partir de definições e escolhas, a fim de contextualizar de que homicídio estamos tratando. 


Em geral as pesquisas que buscam analisar os homicídios de forma populacional o fazem de duas maneiras, ou utilizando os dados das vítimas dessa conduta ou de seus agentes. A primeira maneira é muito mais preponderante, em razão da disponibilidades dos dados e da razoável certeza que se pode ter de que uma pessoa foi vítima do homicídio, em detrimento da dúvida que pode pairar sobre a culpabilidade do agente homicida.

Pesquisas que analisam as populações que foram vítimas de homicídio costumam usar dados oriundos de distritos policiais ou de saúde pública, enquanto pesquisas que analisam os agentes trabalham predominantemente com dados carcerários. (referenciar exemplos de pesquisas de ambos os campos)

Essas considerações são importantes para contextualizar o estudo dos homicídios, uma vez que, a depender do enfoque da pesquisa sobre a vítima ou o agente causador, as variáveis que marcam a ocorrência deste evento tendem a variar. Por exemplo, uma pesquisa que visa observar os comportamentos do agente em relação ao uso de droga deve focalizar o papel da substância como o "gatilho da ação"; por outro lado, sob o viés da análise do comportamento da vítima o uso de uma substância tende a ser vista como um marcador de vulnerabilidade. As duas perspectivas podem ser igualmente válidas.

Este livro se orienta pela perspectiva da vítima, seus familiares, amigos ou pessoas próximas.

Outro ponto importante nos homicídios diz respeito a como categorizar, ou ainda, fragmentar a conduta em diversos subgrupos, buscando identificar suas causas. Enquanto no campo do Direito pode-se classificar os homicídios a partir do grau de culpabilidade do agente, para fins epidemiológicos a desagregação deve se dar visando identificar as causas do desfecho. Esta não é uma tarefa fácil. Uma opção poderia ser adotar a classificação proposta no próprio declaração de óbito de ocorrência, determinada pela CID-10, que especifica a agressão em razão do mecanismo utilizado (arma de fogo, objeto contundente, cortante, etc). Dessa maneira, as causas dos homicídios seriam agrupadas segundo a forma como o corpo foi violado, lesionado. Essa opção não é satisfatória, na medida que abre mão do caráter complexo e socialmente determinado da violência levantado no início deste livro.

Desta forma, adota-se, para fins do presente trabalho, inicialmente, a classificação proposta por G. Feltran, que, ao observar a relação dos eventos com as dinâmicas próprias do mundo do crime e outras motivação para os homicídios, a nível nacional, propõe a divisão dos homicídios em 5 grupos\citep{ABSP2019}:

\begin{itemize}
    \item \textbf{GRUPO 1}: mortes internas ao mundo do crime e às suas redes próximas;
    \item \textbf{GRUPO 2:} mortes ocorridas na guerra entre as polícias e o mundo do crime;
    \item \textbf{GRUPO 3:} feminicídios, ou seja, violência letal contra indivíduos com identidade feminina de gênero;
    \item \textbf{GRUPO 4:} latrocínios, ou seja, as mortes da vítima em situações de roubo;
    \item \textbf{GRUPO 5:} homicídios de LGBTs. 
\end{itemize}

A classificação de Feltran se mostra oportuna, principalmente por duas razões. Em primeiro lugar pela perspectiva antropológica do autor, que inspira, em grande parte, a utilização da autópsia verbal. Em segundo lugar por se basear nos resultados de pesquisa de um grande centro urbano com dinâmicas parecidas com a cidade de Campinas. Ainda que as diferenças entre as cidades sejam significativas, chama a atenção que a mesma dinâmica de queda dos homicídios observada na Capital repetiu-se em Campinas. A explicação de Feltran acerca do fenômeno - acompanhada por outros pesquisadores - é a que destaca o relevante o papel das organizações criminosas, ou ainda do mundo do crime, sobre a dinâmica dos homicídios na cidade.

Ressalta-se, por fim, que em relação aos feminicídios, homicídios de mulheres motivados por razão de gênero, a categorização proposta será diferente, na medida que este evento guarda peculiaridades significativas em relação aos homicídios masculinos.

\section{Resultados}

Conforme mencionado, 153 moradores de Campinas foram assassinados em 2019, totalizando um coeficiente de mortalidade padronizado de 16,9 homicídios para cada 100 mil moradores de Campinas. Nesse total se incluem 11 feminicídios.

Dentre as vítimas, 135 (88,2\%) eram homens e 18 (11,8\%) eram mulheres, o que resultou em um coeficiente padronizado de mortalidade de 26,6 e 7,8 óbitos para cada 100 mil homens e mulheres, respectivamente. Quarenta e quatro (28,8\%) mortos eram de cor branca ou amarela, enquanto 109 (71,2\%) eram pardos ou pretos. Esse dado contrasta muito com a distribuição de raça/cor da população de Campinas, onde cerca de 2/3 são brancos 1/3 são não brancos. De fato, no grupo controle, que representa a população da cidade, 66,2\% eram
brancos ou amarelos, enquanto 33,8\% eram pardos ou pretos.

A Tabela \# mostra estatísticas obtidas 6 ao analisarem-se os dados na forma de um estudo caso-controle \citep{rothman2008modern}, sendo as vítimas de homicídio o grupo de casos e a população viva amostrada o grupo controle.


Pode-se interpretar esses resultados da seguinte forma:

\begin{itemize}
    \item Um aumento de 0,01 no IDHM de uma determinada região está associado a uma redução de 6\% no risco de ser vítima de homicídio para seus moradores.
    \item Um aumento de um ano de escolaridade está associado a uma redução de 13\% no risco de ser vítima de homicídio.
    \item O envolvimento com o crime no último mês está associado a um aumento de 19,4 vezes no risco de ser vítima de homicídio.
    \item Ter sofrido ameaça de morte no último mês está associado a um aumento de 9,7 vezes no risco de ser vítima de homicídio.
    \item Homens têm 5,5 vezes mais risco de serem vítimas de homicídio do que mulheres.
    \item Trabalhadores informais têm 2,8 vezes mais risco de serem vítimas de homicídio do que trabalhadores formais.
\end{itemize}

Conforme pormenorizado na seção Recorte Territorial, o território de Campinas foi classificado em três áreas, não necessariamente contínuas, de acordo com tercis da distribuição dos valores de IDHM das 187 Unidades de Desenvolvimento Humano definidas na cidade, aqui chamadas de regiões pobre, média e rica. A Figura \# mostra a distribuição dos locais de ocorrência de homicídios nessas regiões, ilustrando a desigualdade da distribuição dessas mortes na cidade. Levando-se em consideração as populações dessas regiões, os coeficientes padronizados de homicídios foram 32,2; 14,4 e 6,0 por cem mil habitantes, respectivamente para as regiões pobre, média e rica. Assim, o risco médio de ser assassinado nas regiões pobre e média, em relação à região rica da cidade (medida conhecida como risco relativo), foi estimado em 5,4 e 2,4 respectivamente. Em outras palavras, o risco de um campineiro morrer assassinado na região pobre é mais do que cinco vezes maior do que o risco de ele ser assassinado na região rica.

A Tabela \# apresenta coeficientes de mortalidade padronizados e estimativas de risco relativo de homicídio segundo regiões da cidade e sexo.

Um modo mais sofisticado de enxergar essa desigualdade é visualizar a distribuição do risco relativo espacial de homicídio, conforme ilustra a Figura \#, obtida por meio de um ajuste que tecnicamente é chamado de modelo aditivo generalizado aplicado a um estudo caso-controle espacial \citep{kelsall1995non}\citep{kelsall1998spatial}. Nela é mostrado como varia, palmo a palmo no território de Campinas, o risco de um morador ser assassinado, em relação ao risco médio de homicídio no município. Na figura, o risco aumenta na medida em que se transita de áreas brancas para áreas vermelhas.

\section{Discussão}