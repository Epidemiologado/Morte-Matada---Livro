



\chapter{Acidentes do Trabalho}

\section{Introdução}

Qualquer agressão pontual – seja ela fortuita ou intencional; seja ela decorrente da ação de terceiros, de animais, de fenômenos naturais, de máquinas ou objetos; seja ela causada por agentes químicos, físicos ou biológicos; ou mesmo seja ela uma lesões autoinfligidas – que ocorra durante o trabalho, ou em deslocamentos necessários para seu exercício, é um acidente do trabalho. Eles constituem o maior agravo à saúde dos trabalhadores brasileiros. Diferentemente do que o nome sugere, os acidentes do trabalho não são eventos acidentais ou fortuitos \citep{tsai1991relationship}, mas sim fenômenos socialmente determinados \citep{dwyer2013life}, previsíveis e preveníveis.

Neste livro, as mortes violentas foram primariamente classificadas em cinco grandes grupos. Os acidentes do trabalho permeiam essa classificação, estando presentes em todos esses grupos.

Somente no ano de 2017, em sua consolidação mais atual, o Ministério da Previdência Social do Brasil registrou a ocorrência de 440.914 acidentes de trabalho em todo o território nacional \citep{social2019anuario}. Dentre os atingidos, 2.096 faleceram em consequência do acidente \citep{social2019anuario}. Em sua maior parte, eram trabalhadores jovens e produtivos. Destaque-se que esses dados são notoriamente sub-registrados no Brasil, conforme pesquisadores mostraram ao longo das últimas décadas\citep{possas1987avaliaccao} \citep{lucca1994acidentes}\citep{santana2005acidentes}\citep{cordeiro2006incidencia}.

Tradicionalmente, têm sido apontadas duas causas importantes para essa subnotificação. Por um lado, não existe um sistema único e eficiente que centralize as informações sobre AT no país. Os diferentes sistemas existentes têm efetividade limitada e pouco trocam informações entre si. Por outro lado, o banco de dados mais específico existente, do Ministério da Previdência Social, além de concederem às empresas a prerrogativa de notificar os acidentes ocorridos sob sua responsabilidade, ignora os acidentes de trabalho ocorridos não apenas entre autônomos, funcionários públicos e proprietários, mas também
aqueles que ocorrem no mercado informal da economia brasileira, que em julho de 2019 abrangia 38,7 milhões de brasileiros trabalhando sem nenhum contrato de trabalho, o que equivale a 41,3\% da força de trabalho do país na ocasião \citep{globo2020}. E esse número segue crescendo desde então.
    
Entretanto, há muito tempo o crescimento da violência no Brasil e a dificuldade em identificar seus reflexos sobre a população trabalhadora têm sido apontados como fatores relevantes tanto para a ocorrência de acidente de trabalho, como também para seu sub-registro no país. De fato, diversos estudos realizados nas últimas duas décadas apontaram a crescente participação de homicídios, latrocínios, sequestros, conflitos com criminosos, com policiais, com colegas de trabalho, com clientes e usuários, suicídios, e mesmo o impacto de balas perdidas, como desencadeadores de acidentes do trabalho fatais \citep{oliveira1997acidentes, mendes2003verso, CordeiroTrab, machado1994acidentes, waldvogel1999acidentes, paes2002assaltantes, hennington2004trabalho, santana2013fatal, drumond2013avaliaccao, lacerda2014acidentes, cordeiro2017violencia}. Muitas dessas mortes não são reconhecidas como acidentes do trabalho. Este estudo tenta chamar a atenção para a relação entre violência e acidente do trabalho, e o impacto dessa relação na subnotificação desses agravos.

\section{Método}

Os acidentes de trabalho foram aqui definidos, conforme a Secretaria de Vigilância em Saúde do Ministério da Saúde, como “aqueles que ocorrem no exercício da atividade laboral, ou no percurso de casa para o trabalho” \citep[pg. 753]{ministerio2019guia}. O acidente de trabalho fatal foi definido como “aquele que leva a óbito imediatamente após sua ocorrência ou que venha a ocorrer posteriormente, a qualquer momento, em ambiente hospitalar ou não, desde que a causa básica, intermediária ou imediata da morte seja decorrente do acidente”\citep[pg. 753]{ministerio2019guia}.

A partir das informações coletadas nas autópsias verbais, bem como nas fontes complementares utilizadas, as mortes violentas ocorridas foram reclassificadas dicotomicamente como “acidentes do trabalho fatais” ou “outras mortes violentas”.

Para os propósitos desta análise, os acidentes de trabalho fatais encontrados foram classificados como:

\begin{itemize}
    \item Acidentes de trabalho por crime (AT/cr): decorrentes primariamente de ato criminoso, intencional ou não, contra o trabalhador;
    \item Acidentes de trabalho estritos (AT/es): excluída a classe anterior, são aqueles originados primariamente na execução de atividades laborais;
    \item Acidentes de trabalho no transporte (AT/tr): excluídas as classes anteriores, são aqueles decorrentes primariamente de colisões, atropelamentos ou quedas de veículos motorizados ou não no trânsito, bem como desequilíbrios ou quedas do trabalhador durante locomoção a pé;
    \item Outros acidentes de trabalho (AT/ou): acidentes não enquadrados em nenhuma classe anterior
\end{itemize}

O Fluxograma abaixo ilustra o algoritmo utilizado nessa classificação.

\section{Resultados}

Dentre as 553 mortes violentas encontradas neste estudo, 64 foram classificadas como acidente de trabalho, o que correspondeu a uma mortalidade proporcional de 11,6\%. Chama atenção a magnitude da incidência de acidente de trabalho fatal em Campinas. A cada 9 mortes violentas acontecendo entre moradores da cidade, uma pode ser categorizada como acidente de trabalho. A maioria das vítimas trabalhava no setor de serviços. A Tabela \# apresenta a distribuição das 27 ocupações dos 64 trabalhadores falecidos, segundo Grandes Grupos da Classificação Brasileira de Ocupações \citep{MTE2016CBO}.

Tabela \#: Classificação das ocupações dos trabalhadores mortos em 2019, residentes de Campinas, SP, segundo Grandes Grupos da Classificação Brasileira de Ocupações.

Levando-se em consideração o tamanho da população sob risco de sofrer acidente de trabalho no município \citep{IBGE2020pop}, em 2019 foram identificados 13,9 acidentes de trabalho fatais para cada 100 mil moradores economicamente ativos. Para o sexo masculino, esse coeficiente foi de 24,7 óbitos por 100 mil trabalhadores. A Figura \# apresenta a distribuição espacial dos locais de ocorrência dos acidentes de trabalho fatais identificados em 2019.

Figura \#: Distribuição espacial dos locais de ocorrência dos acidentes de trabalho fatais, Campinas 2019.

Quanto ao sexo, 57 (89,1\%) trabalhadores eram homens e 7 (10,9\%) eram mulheres. A média e a mediana de idade ao morrer foram 41 e 38 anos, respectivamente. O trabalhador mais jovem tinha 19 anos e o mais velho 82 anos. A escolaridade média e mediana do grupo, medida em anos completos, foi de 9 e 11 anos, respectivamente, variando entre 0 e 16 anos. Com relação à regulamentação do trabalho no momento do óbito, encontrou-se 38 (59,4\%) trabalhadores formais e 26 (40,6\%) informais. Quanto à cor, 37 (57,8\%) eram amarelos ou brancos e 27 (42,2\%) eram pardos ou pretos.

A Tabela \# apresenta a classificação dos acidentes de trabalho fatais analisados. Nela observa-se que cerca de um quarto deles ocorreram estritamente ligados à atividade laboral, enquanto os outros três quartos distribuíram-se nas categorias crime, transporte e outros.

Chamam a atenção algumas associações encontradas. Constatou-se que 52 (81,3\%) vítimas trabalhavam sós (desacompanhadas) no momento do acidente; 49 (76,7\%) morreram na rua; 16 (25,0\%) sofreram seus acidentes em zonas de alta criminalidade; 8 (12,5\%) trabalhadores foram vitimados entre 21 e 24 horas
e 7 (10,9\%) entre 5 e 8 horas.

A Tabela \# mostra estatísticas obtidas ao analisarem-se os dados na forma de um estudo caso-controle \citep{rothman2008modern}, sendo todas as vítimas de acidentes do trabalho fatal o grupo de casos e a população trabalhadora viva amostrada o grupo controle.

Tabela \#: estatísticas obtidas no estudo caso-controle tendo como casos os acidentes do trabalho fatais e como controles a população trabalhadora viva amostrada.


Pode-se interpretar esses resultados da seguinte forma:
\begin{itemize}
    \item Um aumento de 0,01 no IDHM de uma determinada região dentro de Campinas está associado a uma redução de 4\% no risco de ser vítima de acidente do trabalho fatal para moradores dessa região.
    \item Um aumento de um ano de escolaridade está associado a uma redução de 12\% no risco de ser vítima de acidente do trabalho fatal.
    \item O envolvimento com a produção ou distribuição de drogas ilícitas no último mês está associado a um aumento de 12,41 vezes no risco de ser vítima de acidente do trabalho fatal.
    \item Homens têm 9,21 vezes mais risco de serem vítimas de acidente do trabalho fatal do que mulheres.
    \item Cada aumento de uma hora na jornada semanal de trabalho está associado a um aumento de 3\% no risco de ser vítima de acidente do trabalho fatal.
\end{itemize}



A Figura \# apresenta a distribuição espacial do risco de ocorrência de um acidente do trabalho fatal no território de Campinas, salientando uma nítida área de proteção no centro da cidade, e áreas de risco aumento ao sudeste e nordeste do município.

A Figura \# apresenta também a distribuição espacial do risco de ocorrência de um acidente do trabalho fatal, mas, diferentemente da figura anterior, restrito apenas aos eventos ocorridos em espaços públicos (ruas, praças e rodovias).

\section{Discussão}

A Lei no 6.367, promulgada em 19 de outubro de 1976 pelo General Presidente Ernesto Geisel, definiu juridicamente o AT como “aquele que ocorrer pelo exercício do trabalho a serviço da empresa”\citep{lei636776}. Embora essa lei considere a possibilidade de ocorrência de AT sem vínculo estrito com a atividade de trabalho realizada pelo trabalhador no momento do ocorrido, esse cenário parece ter sido negligenciado em documentos oficiais que se seguiram. Posteriormente, o Ministério da Previdência Social categorizou os AT como típicos e de trajeto. Os primeiros foram definidos como decorrentes da característica da atividade profissional desempenhada pelo acidentado \citep{MPAS2006anuario}. Já os de trajeto seriam os acidentes ocorridos no trajeto entre a residência e o local de trabalho \citep{MPAS2006anuario}.

De lá para cá, muita coisa mudou no Brasil, dentre elas, a violência urbana. A partir dos anos 1970, a violência letal mudou substancialmente seu perfil. Tome-se, como exemplo, a cidade de São Paulo – paradigmática do que ocorreu nas grandes cidades brasileiras e para a qual a Fundação SEADE \citep{SEADE2017estat} disponibiliza uma longa série histórica de dados sobre mortalidade – onde o coeficiente de mortalidade por homicídios esteve sempre abaixo de cinco casos por 100 mil habitantes desde o início do século XX até os anos de 1960. À essa época, a maioria dos homicídios ocorria dentro das residências e decorriam, predominantemente, de questões privadas envolvendo machismo e patriarcalismo no núcleo familiar. Na década de 1970, o coeficiente de mortalidade por homicídios passou a variar entre cinco e dez por 100 mil, atingindo 65 homicídios por 100 mil habitantes em 1999. No ano 2000, 69\% das vítimas foram assassinadas em vias públicas, e menos de 10\%, em suas residências \citep{gawryszewski2000mortalidade, gawryszewski2002homicidios}. A partir dos anos 1970, em São Paulo e no restante do país, cada vez mais os homicídios tornaram-se instrumento de disputa e controle territorial por meio da ação, nas periferias das regiões metropolitanas e regiões de fronteira com outros países, de grupos paramilitares de extermínio (formados por policiais civis e militares), bandos de segurança privada (“justiceiros”, muitas vezes, patrocinados por comerciantes e pequenos industriais locais) e organizações clandestinas envolvidas com o tráfico e a comercialização de drogas e armamentos (facções criminais). Ironicamente, no início da conformação desse cenário de violência demográfica, social e espacialmente especificada, a ideologia dos grupos paramilitares e justiceiros era a da “defesa do trabalhador contra o bandido”.

Também a década de 1970 marcou o início de importantes transformações no mundo do trabalho. Nas economias centrais, e depois migrando para as regiões industrializadas e dependentes dos países ditos em desenvolvimento, inicia-se um quadro de crise estrutural do capitalismo \citep{meszaros1995beyond, polanyi2002rethinking}. Entre muitas repercussões, essa crise fez com que o capital implementasse um importante processo de reestruturação visando recuperar seu ciclo produtivo, o que afetou fortemente o mundo do trabalho(81). A organização da produção passa a se dar cada vez menos utilizando o trabalho estável e cada vez mais às custas de diversificadas formas de trabalho precarizado. A relação empregatícia padrão, caracterizada por emprego por um patrão, em tempo integral, baseado nas premissas do empregador, de longo prazo e com benefícios e seguridades contratuais, declina no mercado de trabalho(82, 83). As novas unidades de produção utilizam cada vez mais horas extras, trabalhadores temporários, em tempo parcial e subcontratados. Criou-se, de um lado, em escala minoritária, o trabalhador “polivalente e multifuncional” da era informacional, sob maior demanda e pressão das empresas \citep{polanyi2002rethinking}. De outro lado, expandiu-se um grande contingente de trabalhadores com baixa qualificação ocupacional e baixa remuneração, exemplificados, entre outros, por trabalhadores ambulantes como faxineiros, entregadores, mensageiros, motoboys, vigilantes, vendedores, prestadores de serviços fazendo as mais diversas tarefas pontuais (“bicos”), adolescentes lavando carros, entretendo motoristas de automóveis particulares, limpando sapatos ou mesmo trabalhando na prostituição e no comércio de drogas ilícitas \citep{meirelles1998vida, myers1988alternative}. A questão espacial assume importante papel na crise estrutural mencionada \citep{botelho2000fordismo}. Dentre tantos reflexos, trabalhadores informais tendem a ter maior mobilidade espacial durante o exercício de suas atividades. Nesse novo contexto, os limites dos ambientes de trabalho se tornaram mais e mais tênues. Alterou-se a topografia do risco. Os tradicionais ambientes fabris, que em décadas anteriores concentravam a maioria das ocorrências dos agravos à saúde dos trabalhadores, passaram cada vez mais a ceder lugar para o “espaço da rua” como sede de acidentes do trabalho \citep{machado1994acidentes, waldvogel1999acidentes, hennington2004trabalho}. Obviamente, o grande aumento da violência como instrumento de controle territorial e a ocupação das ruas por enorme contingente de trabalhadores precarizados tiveram seu impacto no mundo do trabalho, contribuindo, de maneira importante, para a conformação do perfil de morbimortalidade dos trabalhadores brasileiros nas últimas décadas. Entretanto, a classificação dos acidentes de trabalho acima referida não evoluiu nesse período. Os conceitos de AT típico e AT de trajeto criados nos anos 1970 estão consagrados nos meios governamentais, jurídicos e também, deve-se ressaltar, entre os estudiosos do campo da Saúde do Trabalhador.

Gostaríamos de argumentar contra a continuidade da utilização dessa nomenclatura. Esses termos induzem, pelo menos no senso comum, a uma concepção desatualizada, e mesmo equivocada. Os bons dicionários da língua portuguesa definem típico como aquilo que normalmente acontece, que é característico, que serve de modelo. O AT não pode ser aceito como típico. Além disso, admitir que o AT decorre (apenas) da característica da atividade profissional desempenhada pelo acidentado, como define o Ministério da Previdência Social, naturaliza o AT, limitando seu reconhecimento, e mesmo atribuindo-o, a situações mais ou menos esperadas e restritas ao desempenho de funções laborais. Se considerarmos – como deve ser – que o AT é qualquer agressão pontual que ocorra durante o trabalho, a categorização típico deixa de lado a maioria dos acidentes sofridos pelo trabalhador. Também a categoria de trajeto não é adequada porque, com frequência, é aplicada erroneamente, não especificando a natureza do acidente. No trajeto, o trabalhador sofre não apenas acidentes circunscritos à sua locomoção, como também acidentes estritamente decorrentes de sua atividade laboral, bem como ações intencionalmente criminosas. Além disso, a utilização da categoria de trajeto não raramente leva a intermináveis discussões de interesse estritamente pecuniário sobre a natureza do trajeto, a habitualidade do trajeto, alterações do trajeto, interrupções intencionais do trajeto etc.

O crescimento da violência urbana no Brasil e a dificuldade em identificar seus reflexos sobre a população trabalhadora implicam tanto no aumento da incidência de AT como também em seu sub-registro no país. Nos dias atuais, permanece relevante o estudo da morbimortalidade decorrente de AT, cuja gravidade é apenas tangenciada pelas estatísticas oficiais disponíveis, conforme já demostraram tantos estudos. A classificação típico/trajeto não capta os reflexos das mudanças no mercado de trabalho em décadas recentes, no contexto do aumento da violência urbana vivenciado. Ao contrário, é mais um elemento a dificultar o real dimensionamento dos AT no Brasil. Não bastasse a ausência de informações sobre acidentes do trabalho fatais que o alijamento de enorme parcela da população trabalhadora do sistema de notificação de AT do Ministério da Previdência Social traz. Não bastasse a subnotificação de acidentes do trabalho fatais, mesmo para a parcela de trabalhadores que oficialmente fazem jus a esse reconhecimento, devido a falhas de notificação do sistema. Também os reflexos da violência urbana incidindo sobre o trabalhador em sua jornada laboral geralmente não são identificados pela classificação típico/trajeto como AT, sendo classificados como violência comum – termo que por si só denuncia a gravidade e a banalização da violência em nosso meio – e assim “contribuindo para a invisibilidade das situações adversas de trabalho responsáveis pela sua ocorrência.” \citep[p. 128]{lacerda2014visibilidade}

O estudo do trabalho e suas repercussões na vida do trabalhador deve implicar não apenas na análise de questões diretamente ligadas ao trabalho, mas também na sua articulação com toda a sociedade da qual é produto e com a qual interage \citep{mendes2003verso}. Refutar a classificação oficial e categorizar os At como crime, estrito, transporte e outros tem duas intenções. Por um lado, ressaltar que a organização do trabalho e os modos como o trabalhador é levado a desenvolver suas atividades laborais são apenas uma parte importante do problema. Por outro lado, dar maior visibilidade aos reflexos das mudanças no mercado de trabalho em décadas recentes – com a expansão do setor de serviços e a maior exposição do trabalhador ao ambiente da rua – no contexto da violência urbana brasileira, expresso pela criminalidade e os conflitos por ela gerados, pela agressividade do trânsito, pelo aumento da incidência do suicídio. Hoje, qualquer ação que vise diminuir a ocorrência de acidentes do trabalho precisa contemplar essa realidade.

A quantidade de acidentes de trabalho encontrada em Campinas no ano 2019 é, por si só, alarmante; principalmente porque refere-se a eventos absolutamente evitáveis, que expressam negligência e injustiça social \citep{santana2006acidentes}.
...continua..