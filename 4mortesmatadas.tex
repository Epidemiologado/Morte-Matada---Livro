\chapter{Mortes Matadas}

A população residente de Campinas somava 1.167.192 almas em 01/07/2019, segundo projeção feita pela Fundação SEADE \citep{SEADE2020}. Neste ano, faleceram 7.459 residentes em decorrência das mais variadas causas, de acordo com os registros da Secretaria Municipal de Saúde de Campinas (SMS)\citep{SMS2020}. Em números absolutos, a principal causa de óbitos em 2019 foram as doenças do aparelho circulatório, que contabilizaram 2.254 óbitos. Em seguida vieram as neoplasias, com 1.623 óbitos, e as doenças do aparelho respiratório, com 872 óbitos. Em quarto lugar apareceram as mortes decorrentes da violência, aquelas chamadas pela Organização Mundial da Saúde de mortes por causas externas, com 606 óbitos.

Dentre as 606 mortes violentas registradas oficialmente pela SMS, 31 (5,1\%) não puderam ser investigadas. Essas perdas decorreram da não localização de parentes e amigos das vítimas, quer por inexistência de endereço domiciliar 10(falecidos em situação de rua), quer por imprecisões e incorreções do endereço; bem como também devido a recusas de familiares/amigos em participar da investigação. Após analisarem-se as 575 mortes restantes, 22 foram excluídas da análise porque, baseando-se nas autópsias verbais e nos laudos de necropsia referentes a esses óbitos, concluiu-se que se tratavam de mortes naturais ou óbitos de não residentes da cidade. Assim, os resultados abaixo apresentados referem-se a 553 mortes violentas incidentes entre moradores de Campinas no ano de 2019.

Proporcionalmente, as mortes violentas corresponderam a 7,8\% de todos os óbitos de Campinas em 2019. O coeficiente de mortalidade por causas externas, padronizado pela distribuição etária e de sexo da população brasileira em 2010, igualou 39,5 óbitos para cada 100 mil habitantes da cidade, sendo 61,3 e 19,7 mortes para cada 100 mil homens e mulheres, respectivamente.

Para esta análise, as mortes foram reunidas em quatro grupos: homicídios (CID X85.0 a Y35.7, exceto Y06.0 a Y07.9), quedas (CID W00.0 a W19.9), mortes no transporte (CID V01.0 a V99.9) e suicídios (CID X60.0 a X84.9). As mortes decorrentes de queimaduras, afogamentos, choques elétricos, envenenamentos, entre outras não especificadas acima, foram agrupadas como “outras” e não analisadas devido ao seu relativo pequeno número. A Figura \# mostra a distribuição espacial dos locais onde os eventos violentos que geraram essas mortes ocorreram.

A Tabela \# mostra a distribuição das mortes segundo grupo e sexo. Nela observa-se que os grupos mais incidentes foram os homicídios (153 óbitos) e as quedas (151 óbitos), seguidos das mortes no transporte (93 óbitos) e os suicídios (83 óbitos). Em todos os grupos de mortes violentas aqui apresentados, o número de homens superou o de mulheres.

A distribuição etária dos óbitos variou entre os grupos de mortes analisados. A mediana da idade ao morrer foi igual a 32, 36, 38 e 79 anos, respectivamente para os homicídios, mortes no transporte, suicídios e mortes por quedas. A Figura \# apresenta a distribuição dessas idades, salientando a diferença entre a estrutura etária dos que faleceram devido a quedas, em relação a todos os outros grupos de óbito.

A seguir, essas mortes serão analisadas pormenorizadamente, segundo grandes grupos e contextualizadas