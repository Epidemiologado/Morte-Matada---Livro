\chapter{Mortes Violentas}

O Sistema de Informações de Mortalidade, do Ministério da Saúde, contabilizou a morte de aproximadamente cinco milhões de brasileiros\footnote{Precisamente, 4.663.489 de pessoas mortas por causas externas segundo dados do DATASUS} – do início de sua operação em 1979 à sua última atualização em 2018 – decorrentes das chamadas causas externas de mortalidade \citep{DataSUS2020}. Destaque-se que essas mortes não se distribuíram ao acaso. Atingiram, predominantemente, a população masculina, jovem, de baixa renda, preta e parda, moradora das periferias dos grandes centros urbanos e de pequenas cidades das regiões de fronteira do Brasil com outros países. Destaque-se, sobretudo, que essas mortes eram evitáveis.

Morte por causa externa é um eufemismo, impregnado nas estatísticas de saúde desde o final do século XIX, para designar os óbitos decorrentes da violência. Em 1855, William Farr apresentou ao 2º Congresso Internacional de Estatística, realizado em Paris, uma classificação de causas de morte reunidas em cinco grandes grupos, sendo um deles designado como "doenças ou mortes violentas"\citep{laurenti1991analise}. Esta iniciativa influenciou a Classificação de Causas de Morte de Bertillon, de 1893, que, por sua vez, foi a base para o desenvolvimento da Classificação Internacional de Causas de Morte e de Doenças (CID) durante todo o século XX. De sua primeira (1900) à nona (1975) revisão, o número de categorias de mortes violentas cresceu de forma acentuada, estando elas atualmente agrupadas nos capítulos XIX e XX da 10a Revisão da Classificação Internacional de Doenças, publicada em 1994 \citep{organizaccao2000classificaccao}. Essas classificações serviram, ao longo do século XX e início do XXI, para a análise de perfis de mortalidade de diferentes locais e nações, bem como comparações entre elas. Em junho de 2018 a Organização Mundial da Saúde (OMS) divulgou a 11a Revisão da CID, ainda não implantada no Brasil.

Entretanto, durante esse processo o termo violência – que remete a causalidade de ordem social, econômica, política, ambiental, espacial - foi substituído por \textit{causa externa}, de caráter neutro.

As mortes relacionadas à violência são brutalidades que ceifam vidas humanas e trazem sofrimentos às pessoas próximas das vítimas. Para os familiares e amigos daqueles que morreram, vítimas ocultas da agressão \citep{soares2006vitimas}, a violência traz ainda uma série de consequências nocivas, tais como o afastamento e o enfraquecimento dos laços familiares, a diminuição do desempenho no trabalho, o desinteresse por momentos de lazer e dificuldades financeiras \citep{costa2017repercussoes}.

% atualizei até a pasta dos pdfs dos capítulos no Morte Matada.

A violência consiste de ações de indivíduos, grupos, classes, nações que ocasionam a morte de outros seres humanos ou que afetam sua integridade física, moral, mental ou espiritual\citep{minayo1997violencia}. A violência afeta a saúde porque representa um risco maior para a realização do processo vital humano: ameaça a vida, altera a saúde, produz enfermidade e provoca a morte como realidade ou como possibilidade próxima \citep{franco1990violencia}.

A Organização Mundial da Saúde, em seu \textit{Relatório Mundial sobre Violência e Saúde}, de 2002, definiu a violência como “o uso da força física ou do poder real ou em ameaça, contra si próprio, contra outra pessoa, ou contra um grupo ou uma comunidade, que resulte ou tenha qualquer possibilidade de resultar em lesão, morte, dano psicológico, deficiência de desenvolvimento ou privação”\citep{organizaccao2002relatorio}. É importante ressaltar que tal definição não contempla as violências não intencionais, aquelas onde não havia a intenção explícita de matar ou gerar algum dano, como por exemplo, os acidentes de trânsito, os acidentes de trabalho, as quedas, os choques, dentre tantos outros. Neste texto, salvo quando anotado de forma diversa, chamaremos de violência todas as situações, intencionais ou não intencionais, que impliquem em ofensas ao corpo físico e/ou mental do ser humano.

A violência social no Brasil é um fenômeno complexo, cujas raízes se aninham nas desigualdades sociais que o país criou ao longo de seu desenvolvimento. Ela se expressa tanto em conflitos nas relações entre indivíduos, quanto entre grupos; tanto em organizações da sociedade civil, quanto em instituições do Estado; tanto no meio rural quanto no urbano. O Guia de Vigilância em Saúde, do Ministério da Saúde, refere que “a violência possui causas múltiplas, complexas e correlacionadas com determinantes sociais e econômicos – desemprego, baixa escolaridade, concentração de renda, exclusão social, entre outros –, além de aspectos relacionados aos comportamentos e cultura, como o machismo, o racismo, o sexismo e a homofobia/lesbofobia/transfobia” \citep{ministerio2019guia}. As mortes violentas se alternaram ao longo das últimas décadas entre a segunda e a quarta maior causa de morte, em números absolutos, no quadro geral de mortalidade brasileiro, assumindo com folga o primeiro posto entre os jovens. Ainda assim, apenas ao final dos anos 1980 a discussão da violência começa a encontrar espaço na agenda da Saúde Pública \citep{minayo1994violencia}.

A violência social não é um fenômeno específico da área da saúde, mas afeta a saúde das populações de maneira marcante e inequívoca. “Tudo o que significa agravo e ameaça à vida, às condições de trabalho, às relações interpessoais, à qualidade da existência, faz parta da Saúde Pública.”\citep{minayo1999possivel} Já em 1993, a Organização Pan-Americana de Saúde alertava que “a violência, pelo número de vítimas e a magnitude de sequelas orgânicas e emocionais que produz, adquiriu um caráter endêmico e se converteu num problema de Saúde Pública em vários países.”\citep{OPASwashington1993}

Serviços de saúde “são a encruzilhada para onde convergem todos os corolários da violência”\citep{OPASwashington1993}. Além do sofrimento físico e emocional que provoca nas vítimas diretas e em toda a população, a violência sobrecarrega os serviços de emergência, reabilitação, saúde mental, serviço social, Institutos de Medicina Legal, aumentando sobremaneira os custos dos serviços de saúde e deteriorando a qualidade da atenção prestada. 

Além de contabilizar os mortos, profissionais da saúde buscam amenizar os efeitos da violência sobre o corpo e a mente dos indivíduos, cuidando da reparação de traumas físicos em serviços de emergência e reabilitação, bem como abordando impactos psicológicos e psicossociais sobre as vítimas e as pessoas que com elas convivem. Entretanto, cabe também a esses profissionais ultrapassar o papel curativo, atuando na prevenção dos agravos associados à violência.

Os adolescentes e adultos jovens constituem o grupo de idade mais afetado pela violência em suas distintas formas, sendo não apenas vítimas, mas também geradores ou intermediadores. No Brasil, esse grupo populacional tem comprometido cada vez mais sua condição de sobrevivência em decorrência de mortes evitáveis e preveníveis. A ocorrência das causas violentas de morte entre faixas etárias jovens tem chamado a atenção na discussão desse fenômeno. As mortes que ocorrem nesta etapa da vida, de grande potencial criativo e produtivo, além de atingirem as vítimas e o grupo mais próximo a elas, também priva a coletividade de seu potencial econômico e intelectual.

A Organização Mundial da Saúde estima que anualmente ocorram mais de três milhões de mortes violentas em todo o mundo nos dias atuais \citep{world2018world}. Cerca de 1,25 milhão decorre de acidentes de trânsito, dos quais 90\% incidem em países de baixa e média renda, atingindo predominantemente vulneráveis (pedestres, ciclistas e motociclistas), jovens, do sexo masculino \citep{Who2018road}. A segunda maior classe de mortes violentas parece ser os suicídios, com cerca de 800 mil ocorrências anualmente\citep{WHO2019suicide}. A seguir, as mortes causadas por quedas somam 640 mil ocorrências anualmente, atingindo predominantemente a população com mais de 65 anos de países de baixa ou média renda \citep{who2018falls}. Já os homicídios somaram 566 mil ocorrências em todo o mundo no ano 2017, atingindo predominantemente jovens de baixa renda da América Latina e do continente africano \citep{mc2017global}.

A mortalidade por causas externas no Brasil guarda as mesmas características das mortes em nível global em relação à maior incidência no sexo masculino, grupo etário jovem (exceto para as mortes por quedas) e altas cifras concentradas nas periferias das grandes cidades e em cidades pequenas e médias das regiões de fronteira com outros países. A partir da década de 1980 as causas externas ocuparam o segundo lugar no quadro geral de mortalidade em várias regiões do País.

No último relatório das Estatísticas Globais de Saúde, de 2018, o Brasil aparece apresentando o sétimo maior coeficiente de mortalidade padronizado por homicídios das Américas, com 31,3 óbitos por 100 mil habitantes, colocando-se entre os países mais violentos do continente, atrás apenas de Honduras [55,5/100mil], Venezuela [49,2/100mil], El Salvador [46,0/100mil], Colômbia [43,1/100mil], Trinidad Tobago [42,2/100mil], e Jamaica [39,1/100mil] \citep{world2018world}. Países europeus apresentaram nesse ano coeficientes de mortalidade por homicídios substancialmente mais baixos, tais como Inglaterra [1,3/100mil], Portugal [1,2/100mil] e França [0,9/100mil] \citep{world2018world}. Em 2018 foram registrados no Sistema de Informações de Mortalidade do Ministério da Saúde (SIM) 55.834 homicídios no Brasil \citep{DataSUS2020}. A magnitude dos homicídios é bastante heterogênea no país. Em 2017 ela variou de 10,3 óbitos por 100 mil residentes no estado de São Paulo a 62,8 óbitos por 100 mil residentes no Rio Grande do Norte \citep{cerqueira2019atlas}.

No Brasil, o número de mortes por acidente de trânsito registrados no SIM foi de 33.625 3 óbitos em 2018 \citep{DataSUS2020}, correspondendo a um coeficiente bruto de mortalidade de 16,1 por 100 mil habitantes. Entre 2000 e 2013, a taxa de mortalidade dos motociclistas aumentou de 1,5 para 6,0 mortes por 100 mil habitantes\citep{brasil2015saude}.

Ainda segundo o SIM, as mortes por quedas e suicídios incidiram em 15.937\footnote{Dados sujeitos a atualização.} e 12.733\footnote{Dados sujeitos a atualização.} residentes do país, respectivamente, no ano de 2018, correspondendo a coeficientes de mortalidade brutos de 7,6 e 6,1 óbitos por 100 mil habitantes

Na cidade de Campinas foram registrados 13.378 óbitos por causas externas no período de 2000 a 2019, sendo que entre 2000 e 2004 aconteceram 31\% desses óbitos. Quanto ao sexo, a proporção de homens (79,8\%) é maior que nas mulheres (20,2\%). O grupo etário compreendido entre 15 e 34 anos correspondeu a 44,1\% das vítimas, sendo a maior proporção concentrada entre os 15 e 19 anos.

Campinas, desde os anos 1970, é considerada uns dos grandes centros industriais do Brasil \citep{nascimento2016regiao}. Esse atributo trouxe consigo os problemas das grandes aglomerações urbanas, promovendo movimentos migratórios, ampliando a desigualdade e a iniquidade na saúde e gerando expressivos índices de violência \citep{aidar2003impacto, francisco2004distribuiccao}. O termo desigualdade em saúde faz referência às diferenças nos níveis de saúde de grupos socioeconômicos distintos em um sentido descritivo \citep{duarte2002epidemologia}. Já o termo iniquidade tem uma dimensão ética–moral, sendo entendido como diferenças sistemáticas, desnecessárias, visíveis, injustas e evitáveis(23). Refere-se às diferenças nos níveis de saúde de grupos socioeconomicamente distintos, considerando-se injustas com base em um julgamento detalhado de suas causas; adicionando uma avaliação ética ao sistema criador das desigualdades\citep{duarte2002epidemologia, nunes2001medindo}. 

Homicídios, suicídios, feminicídios, atropelamentos e demais lesões no trânsito, quedas, lesões do trabalho, queimaduras, envenenamentos, eletrocussões, afogamentos, soterramentos, lesões por deslizamento de terra ou rompimento de barragem não são acidentais. Não são fortuitos. Não são fatalidades. Não são frutos do acaso. Não são má sorte. Ao contrário, são ocorrências socialmente determinadas, previsíveis e preveníveis. São fenômenos evitáveis porque a violência não é inerente à vida social.
