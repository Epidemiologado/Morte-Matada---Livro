
\chapter{Suicídios}

\section{Introdução}
A palavra suicídio começou aparecer no século V, nesta época São Agostinho focava este fato como uma ação pecaminosa. Mais adiante, na idade média o suicídio foi entendido pela coroa como um crime, já que eles confiscaram os bens de aqueles que se matavam, já no final da idade média quando os médicos começaram a ter um lugar mais privilegiado na sociedade, se inicia olhar ao suicídio como loucura, separando ele da visão da igreja como um ato de pecado. https://site.cfp.org.br/wp-content/uploads/2013/12/Suicidio-FINAL-revisao61.pdfEstes questionamentos morais ao longo da história, todavia se podem observar na atual sociedade capitalista, embora menos marcados que na antiguidade.
Este estudo entende o suicídio como a morte produzida por uma autolesão, mas também reconhece á sociedade como elemento que constitui parte da responsabilidade de este fenômeno. O qual permite entender o suicídio como uma morte evitável que pode ser prevenida com ações de políticas públicas desde múltiplas perspectivas. 
O suicídio classificado no CID10 X60.0 a X84.9, trata-se de um fenômeno complexo influenciado por uma variedade de características a nível populacional e individual (Turecki PG. Turecki G, Brent DA. Suicide and suicidal behaviour.The Lancet.2016;387(10024):1227-1239. Lancet. 2017;387(10024):1227–39.), podendo ter múltiplos fatores determinantes como sexo, depressão, ideação suicida, transtorno psiquiátrico, abuso de substâncias, histórico de violência na infância, violência física/financeira/psicológica, ansiedade, culpa, desamparo, doença terminal, incapacidade de pedir ajuda, isolamento, problemas visuais, baixa autonomia, dependência funcional, desesperança, desvalorização social, fracasso, frustração, descuido com medicação, ideação suicida, plano suicida, distúrbio psiquiátrico entre  outras (de Sousa GS, Perrelli JGA, de Oliveira Mangueira S, de Oliveira Lopes MV, Sougey EB. Clinical validation of the nursing diagnosis risk for suicide in the older adults. Arch Psychiatr Nurs [Internet]. 2020;34(2):21–8. Available from: https://doi.org/10.1016/j.apnu.2020.01.003... Miranda BC, Sánchez MH, Pérez RMG. Mortalidad por suicidio, factores de riesgos y protectores. Rev Habanera Ciencias Medicas. 2016;15(1):90–100.)
Os suicídios constituem um importante problema de saúde pública no Brasil e no mundo. De acordo com a Organização Mundial de Saúde (OMS), cerca de 800 mil suicídios ocorrem a cada ano no planeta, distribuídos desigualmente por todas as faixas etárias, numa proporção de 3 homens para cada mulher (World Health Organization. World Health Statistics 2018: Monitoring Health for the SDGs Sustainable Development Goals [Internet]. 2018. Available from:https://apps.who.int/iris/bitstream/handle/10665/272596/9789241565585-eng.pdf?ua=1) . Esta é a segunda principal causa de morte entre jovens de 15 a 29 anos. Em 2016, o coeficiente de mortalidade por suicídio padronizado por idade era, globalmente, cerca de 11 casos a cada 100 mil habitantes (World Health Organization. Suicide in the world: Global Health Estimates. World Health Organization,Geneva. 2019. p. 32.) . Esse coeficiente varia de cerca de 3 até 30 óbitos por 100 mil habitantes entre os diferentes países, sendo maior naqueles de baixa e média renda (World Health Organization. Suicide in the world: Global Health Estimates.World Health Organization,Geneva. 2019. p. 32.) . No Brasil, segundo dados do Sistema de Informação de Mortalidade (SIM), o coeficiente de mortalidade por suicídio foi 6,1 mortes para cada 100 mil habitantes em 2018 (Sistema de Informações de Mortalidade [Internet]. DATASUS. 2020 [cited2020 May 26]. Available from: http://www2.datasus.gov.br/DATASUS/index.php?area=0205) . Entre homens esse coeficiente foi 9,7 e para as mulheres 2,6 por 100 mil habitantes.) 
A (OMS) informa no ano de 2019, que o principal mecanismo de suicídio ao nível mundial, é a auto intoxicação representando o (20%) dos suicídios, acontecendo em sua maioria com praguicidas, nas áreas rurais de países de renda baixa, outros dos principais métodos ao nível mundial são o enforcamento e as armas de fogo. https://www.who.int/es/news-room/fact-sheets/detail/suicide. Desde o ano 2010 até o 2017 o principal mecanismo de suicídio em Campinas foi o enforcamento, seguido do envenenamento/auto intoxicação, armas de fogo, precipitar-se de lugar elevado, entre outros mecanismos. Segundo o sexo os mecanismos mais comuns em mulheres foram o enforcamento (46,9%), auto intoxicação (20,4%) e precipitar-se de um lugar elevado (14,3%), enquanto aos homens foram o enforcamento (62,1%), arma de fogo e envenenamento/auto intoxicação (10,8%). 
Em 2010, no Município de Campinas a magnitude de manifestações de comportamentos suicidas na população urbana, mostrou que (7%) dos habitantes de 14 anos ou mais já pensaram seriamente em pôr fim à vida; (5%) chegaram a planejar o ato, e (3%) efetivamente tentaram o suicídio (Prevalence of suicidal ideation, suicide plans, andattempted suicide: a population-based survey in Campinas, São Paulo State, Brazil.). Neste mesmo município observou-se que a taxa de mortalidade por suicídio foi de 2,7 por 100.000 habitantes nos anos 1996/1998, já no 2015/2017 atingiu 5,5 por 100.000 habitantes. O aumento do suicídio como problema de saúde pública pode ser observado também quando observamos a taxa de contribuição do suicídio nas causas externas de mortalidade, que em 2000 representava (3,7%) das causas externas, chegando a (10,8%) em 2017. Nota-se também, neste mesmo ano que  mortalidade por suicídio no sexo masculino eram 4,4 vezes maior em comparação com o feminino. (Departamento de vigilância e saúde (DEVISA)-Prefeitura municipal de Campinas. Mortalidade por suicídio. 2019.) . seguindo a tendência de maior incidência de suicídio masculino na população brasileira.
Em 2019 foi sancionada a Política Nacional de Prevenção da Automutilação e do Suicídio, por meio da Lei Federal n.º 13.819,  na procura da promoção da saúde mental, o melhoramento no atendimento, controlando os fatores de risco que influenciam na automutilação, vinculando entidades de saúde, educação, comunicação, imprensa, polícia, entre outras. Ainda que se tenham políticas para a prevenção do suicídio, redirecionadas especificamente com essa nomeação, vale a pena considerar que como o suicídio é determinado por múltiplos fatores, não só se precisa de ações focadas no monitoramento, atendimento e tratamento para prevenir esta causa externa. Também se precisa de criar politicas que valorizem as condições de vida https://site.cfp.org.br/wp-content/uploads/2013/12/Suicidio-FINAL-revisao61.pdf. Que permitam mitigar os grandes problemas da atual sociedade brasileira como a desigualdade que vem acompanhada de falta de oportunidades, desemprego, disponibilidade e acesso livre e econômico de produtos prejudiciais para a saúde, atendimento desde o ponto de vista biomédico entre outras.

\section{Resultados}

Durante o ano de 2019, 83 moradores de Campinas cometeram suicídio. Isso implicou em um coeficiente de mortalidade padronizado de 6,4 óbitos para cada 100 mil moradores.

Com relação ao sexo das vítimas, 64 (77,1\%) eram homens e 19 (22,9\%) eram mulheres, o que resultou em um coeficiente padronizado de mortalidade de 8,8 e 4,1 óbitos para cada 100 mil homens e mulheres, respectivamente. Quarenta
e nove (59,0\%) das vítimas eram de cor amarela ou branca, enquanto as restantes 34 (41,0\%) eram pardas ou pretas.

A Tabela \# mostra estatísticas obtidas ao analisarem-se os dados na forma de um estudo caso-controle, sendo o grupo de casos composto pelas vítimas de suicídio e o grupo controle constituído pela população viva amostrada com idade igual ou superior a 10 anos.

Tabela \#: estatísticas obtidas no estudo caso-controle tendo como casos os falecidos em acidentes de transporte e como controles a população viva amostrada.

Pode-se interpretar esses resultados da seguinte forma:

\begin{itemize}
    \item Um aumento de 0,01 no valor da IDHM em determinada região da cidade está associada a uma redução de 4\% no risco de cometer suicídio nessa região.
    \item Ter sentido medo de sofrer violência por parte de criminosos e/ou policiais está associado a uma redução de 77\% do risco de cometer suicídio.
    \item Ter idade entre 10 e 29 anos está associado a um aumento do risco de cometer suicídio de 6,2 vezes em relação a quem tem idade maior que 64 anos.
    \item Ter idade entre 30 e 49 anos está associado a um aumento do risco de cometer suicídio de 7,9 vezes em relação a quem tem idade maior que 64 anos.
    \item Ter idade entre 50 e 64 anos está associado a um aumento do risco de cometer suicídio de 3,6 vezes em relação a quem tem idade maior que 64 anos.
    \item Homens têm risco de cometer suicídio 6,0 vezes maior do que mulheres.
    \item Exercer trabalho informal está associado a um risco 2,8 vezes maior de cometer suicídio em relação a trabalhadores formais.
    \item Não exercer trabalho remunerado está associado a um risco 4,9 vezes maior de cometer suicídio em relação a quem trabalha remuneradamente.
    \item Ser portador de deficiência física está associado a um risco 3,0 vezes maior de cometer suicídio.
\end{itemize}

A Figura \# mostra a distribuição da moradia das vítimas de suicídio em 2019 segundo as regiões pobre, média e rica da cidade.

A Figura \# apresenta a distribuição espacial do risco de suicídio no município de Campinas, ressaltando a existência de uma área de proteção no centro da cidade.

\section{Discussão}

Pode se observar que apresentaram maior risco de suicídio os homens, que não exercem trabalho remunerado ou que tinham trabalho informal, que apresentam algum tipo de deficiência, na faixa etária de 30-49 anos. Concomitantemente como fator protetor se encontraram a percepção de medo e maior IDHM. Resultados que permitem captar a multifatorialidade do suicídio, deixando clara a importância de prevenir o suicídio por meio de implicações sociais e econômicas, além aspectos clínicos e individuais.
Estes resultados que abrangem várias perspectivas sociais e individuais de uma cidade de grande porte, permitem apoiar uns dos objetivos da Política Nacional de Prevenção da Automutilação e do Suicídio, sendo a determinação de fatores que possam influenciar no final fatal, o qual pode gerar ações orientadas e priorizadas para a execução de uma política mais eficaz e responsiva às necessidades da população. 
O risco aumentado no sexo masculino é na faixa etária de 30-49 anos também foi observado a nível nacional e mundial, segundo dados do sistema de informação de mortalidade no brasil para o ano 2018 os homens apresentaram um maior coeficiente de mortalidade sendo 3,7 vezes maior em comparação com as mulheres. (Sistema de Informações de Mortalidade [Internet]. DATASUS. 2020 [cited2020 May 26]. Available from: http://www2.datasus.gov.br/DATASUS/index.php?area=0205))  en quanto à faixa etária no mesmo sentido, o mais recente relatório do Royal College of Psychiatrists do Reino Unido sobre suicídio em adultos apontou que três quartos das pessoas que se suicidaram eram homens, sendo o suicídio a maior causa de morte de homens com menos de 50 anos. RCPsych. Self-harm and suicide in adults: Final report of the Patient Safety Group. 2020;(July). 
A presença de deficiência física/visual/auditiva/intelectual encontrou-se como marcador de risco. O que atenta para a imprescindibilidade de ações que incluam um adequado desenvolvimento de necessidades fundamentais, tanto de movilidade, como de atividades sociais e cotidianas que permitam um trabalho intersetorial em todos os aspectos da vida, gerando assim uma abrangência e inclusão da  população em risco. É importante também considerar que a dependência, a falta de autonomia, desesperança, desvalorização social, são fatores de risco que podem ser gerados ao redor de pessoas deficientes. (de Sousa GS, Perrelli JGA, de Oliveira Mangueira S, de Oliveira Lopes MV, Sougey EB. Clinical validation of the nursing diagnosis risk for suicide in the older adults. Arch Psychiatr Nurs [Internet]. 2020;34(2):21–8. Available from: https://doi.org/10.1016/j.apnu.2020.01.003... Miranda BC, Sánchez MH, Pérez RMG. Mortalidad por suicidio, factores de riesgos y protectores. Rev Habanera Ciencias Medicas. 2016;15(1):90–100.). estes resultados concordam com um estudo prévio que mostrou que a dor e o comprometimento do neurodesenvolvimento se associa ao desfecho. Stenager E. Somatic diseases and suicidal behaviour. Psychiatr Danub. 2006;18(1):151–151. 
Os resultados nos permitem observar que a desigualdade abrangida pela situação laboral e o IDHM foram um grande fator de risco que pode determinar o final fatal. Isto é, quando menor o desenvolvimento e maior a desigualdade, mais elevadas serão as taxas de suicídio. AMARAL, S. S. Suicídio no RN e sua relação com determinantes espaciais, urbanização, desenvolvimento e outros fatores socioeconômicos. Revista Brasileira de Estudos Regionais e Urbanos, v. 13, n. 2, p. 288-308, 16 out. 2019. Evidenciaram-se uma alteração no perfil epidemiológico das mortes por suicídio. Não mais homem brancos de condições econômicas elevadas correspondem ao grupo de risco que requer maior atenção. A alteração do perfil, marcada por uma assimetria do poder, torna, mulheres, idoso, jovens, pretos e pardos, de menor renda, residentes de localidades menos urbanizadas, mais vulneráveis e precárias como os principais grupos de risco em caso de suicídio.
