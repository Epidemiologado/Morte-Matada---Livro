

\section*{Introdução}

O presente manual tem por objetivo oferecer elementos básicos para os entrevistadores participantes da pesquisa "Trabalho, Violência e Morte" em sua abordagem com os entrevistados em uma pesquisa caso e controle.

\section*{A pesquisa}


\subsection*{Objetivos - Questões da Pesquisa}

% Responder: relações entre os feminicídios, A relação entre o contexto formal e  informal de controle do uso, comércio, produção e distribuição de SPA (chamado de Guerra às Drogas).

1) Determinar a distribuição espacial do risco de morrer por causas externas na cidade de Campinas

2) verificar o impacto de variáveis ecológicas, passíveis de serem avaliadas territorialmente com os desfechos estudados (UDH)

% Acidente de trabalho

3) Verificar a associação entre determinantes sociais contemporâneos relacionados à violência Sistêmica na ocorrência desses desfechos, em especial a Guerra às Drogas e a influência dos "mundos do crime".

4) Avaliar os aspectos gerais da violência contra a mulher no município, com especial enfoque nos feminicídios.
%Projeto Mônica

5) Qualificar os dados adquiridos por meio da comparação com os dados do SIM/Segurança Pública e Sistema de Justiça.

%Projeto Mirla

6) constituir os primeiros passos para a construção de um Sistema de Vigilância Epidemiológica da Violência Letal no Município de Campinas.


\section*{Seções do Questionário}

dat\_ent		
entrevist	1.1. Pesquisador	1.1. Pesquisador

\subsection*{Pesquisa}

A seção pesquisa é a mais curta do questionário, e contém dados que serão utilizados principalmente para controle logístico da entrevista.

\subsubsection{Data da pesquisa}

A data da pesquisa é preenchida automaticamente pelo sistema, no momento em que o pesquisador envia o formulário para o sistema

\subsubsection{Pesquisador}
Os pesquisadores registrados na pesquisa FAPESP são:

\begin{itemize}
    \item Alison Douglas da Silva
    \item Ana Cláudia Alves Martins
    \item Flávia Juliana Rodrigues de Sá Pinheiro de Melo
    \item Lígia Cocenas
    \item Mirla Randy Bravo Fernandez
    \item Mônica Caicedo Roa
    \item Pedro Henrique de Faria
\end{itemize}

O pesquisador Antonio Carlos Bellini Junior também aparece na lista de pesquisadores, e está também vinculado à pesquisa da FAPESP, no entanto não participará efetivamente do campo, ficando responsável pela análise dos dados junto ao judiciário.


\subsection*{Dados de identificação}

\begin{itemize}
    \item Status
    \item Número do SIM (Caso)
    \item Nome do Falecido (Caso)
    \item Nome do entrevistado/controle
    \item Vínculo de parentesco do entrevistado (Caso)
    \item Data de nascimento
\end{itemize}

\subsubsection{Status}

A seleção do Status do evento é o que define a relação de risco de passar pelo desfecho (morte). O caso é qualquer pessoa residente de Campinas que venha a ter falecido por causas externas, aquelas explicitadas no Capítulo XX da Classificação Estatística Internacional de Doenças e Problemas relacionados à Saúde (CID 10). O controle é qualquer pessoa residente de Campinas sob risco de passar por quaisquer dos desfechos elencados no Cap XX. Como não há condição biológica, histórica ou social que impede previamente a ocorrência de mortes violentas na população, qualquer residente de Campinas, em qualquer idade pode ser sorteado para ser controle. 

No entanto, em razão da aplicação de entrevistas e da comparabilidade dos dados com a pesquisa realizada anteriormente, que selecionou os casos de 15 a 65 anos essa faixa etária, considerada economicamente ativa, é a selecionada para ser entrevistada durante a pesquisa.

\subsubsection{Código SIM (Caso)}

A cada declaração de óbito registrada é atribuído um número de registro junto ao Sistema de Informação sobre Mortalidade. Esse número estará disponível na base de dados fornecida pela Secretaria Municipal de Saúde de Campinas.

\subsubsection{Nome do falecido (Caso)}

Embora o nome do Falecido seja informado na declaração de óbito é importante confirma e corrigir esse dado junto aos familiares do falecido, inclusive para verificar possíveis incorreções no preenchimento da declaração.


\subsubsection{Vínculo de parentesco do entrevistado (Caso)}

\subsubsection{Nome do Entrevistado/Controle}

Para os casos, a confirmação do nome do entrevistado (e posteriormente telefone) é importante para posterior contato caso o preenchimento do questionário fique incompleto.

Para os controle, esse dado serve também para identificação.


\subsubsection{Data de nascimento}

Importante para calcular a relação da idade com as outras variáveis, além de padronização dos controles e casos em relação à população fonte.



\subsection{Dados Sociodemográficos}

3.1. Raça/Cor do(a) falecido(a)	3.1. Raça/Cor
3.2. Crença(s) Religiosa(s) do(a) falecido(a)	3.2. Crença(s) Religiosa(s)
3.2.1 Descrever outra religião	3.2.1 Descrever outra religião
3.3. Sexo Biológico do(a) falecido(a)	3.3. Sexo Biológico
3.4. Identidade de Gênero do(a) falecido(a)	3.4. Identidade de Gênero
3.4.1. Descrever outra Identidade de Gênero	3.4.1. Descrever outra Identidade de Gênero
3.5. Orientação Sexual do(a) falecido(a)	3.5. Orientação Sexual
3.5.1. Descrever outra orientação sexual	3.5.1. Descrever outra orientação sexual
3.6. O (a) falecido(a) era Alfabetizado?	3.6. É Alfabetizado?
3.7. Escolaridade do(a) falecido(a)	3.7. Escolaridade
3.8. Anos de Escolaridade do(a) falecido(a)	3.8. Anos de Escolaridade


\subsubsection{Etnia/Cor}


As questões raciais devem ser tidas a partir de um conceito de raça que respeite a subjetividade e as identidades étnicas do entrevistado. O IBGE utiliza cinco categorias: branco, pardo, negro, amarelo e indigena que não incorporam as possíveis identidades raciais que existem dentro dos territórios. Portanto, além da auto-referenciação acerca da cor de pele a ser coletada, os pesquisadores devem estar atentos a outros elementos étnicos que podem surgir durante a autópsia verbal. 


\subsubsection{Crença(s) Religiosa(s)}

Existem dois aspectos da religião que podem ser avaliados: a prática religiosa e a identidade religiosa. No Brasil existe a presença do dito "sincretismo religioso", marcado pela assunção por uma mesma pessoa de elementos de diversas ritualísticas, muitas vezes expressando verdadeiros amálgamas espirituais. A umbanda, religião tipicamente brasileira, é um exemplo da expressão do sincretismo que toma forma, por meio do culto à diversas entidades e a mistura de dogmas presentes em religiões de origem cristã e de matriz africana. Para o entrevistador, essa diversidade pode ser colhida a partir da resposta à questão acerca da declaração de religião, que espera-se seja respondida majoritariamente em torno das religiões cristãs, mas não se pode perder de vista que a fé dos entrevistados pode passar por outros símbolos e lugares, o que deve ser absorvido nas entrevistas da autópsia verbal. 
Por outro lado, a prática religiosa, isto é, o quanto a religião de fato pode interferir na vida da pessoa e nas suas escolhas também deve ser avaliada, uma vez que apenas a declaração formal acerca da fé pouco tem a dizer sobre o impacto dessa crença na vida do sujeito e nas percepções acerca dos fenômenos a serem estudados.

\subsubsection{Sexo biológico}

\subsubsection{Identidade de Gênero}

\subsubsection{Orientação Sexual}

\subsubsection{Alfabetização}

\subsubsection{Escolaridade}

\subsubsection{Anos de Escolaridade}

\subsection{Informações sobre o Trabalho}

4.1. O(a) falecido(a) estava trabalhando?	4.1.Está trabalhando?
4.1.1. Se não, qual era a atividade ou fonte de renda do(a) falecido(a)?	4.1.1. Se não, qual sua atividade ou fonte de renda?
4.1.1.2 Descrever outra atividade	4.1.1.2 Descrever outra atividade
4.1.2. Em quantos locais o(a) falecido(a) trabalhava?	4.1.2. Em quantos locais trabalha?
4.1.3. Qual era a ocupação Principal do(a) falecido(a)?	4.1.3. Ocupação Principal
4.1.4. Selecione o Grande Grupo de Ocupação (CBO)	4.1.4. Selecione o Grande Grupo de Ocupação (CBO)
4.1.5. Em seu trabalho principal, qual era seu vínculo?	4.1.5. Em seu trabalho principal, qual seu vínculo?
4.1.6. Quantas horas por semana trabalhava?	4.1.6. Quantas horas por semana está trabalhando?

\subsubsection{Dados acerca do trabalho}

\subsubsection{Local de Trabalho}

\subsubsection{Grande Grupo de Ocupação}

\subsubsection{Vínculo empregatício}

\subsubsection{Horas por semana trabalhava}

\subsection{Dados da Morte - Autópsia Verbal}



5.1 A morte ocorreu em decorrêcia do exercício de atividade laboral ou no trajeto de casa para o trabalho?
5.1.1. Foi decorrente de ato intencional ou não-intencional?
5.1.2. Foi originário primariamente de atividade laboral?
5.1.3. Foi decorrente de eventos no trânsito (atropelamento, quedas no trânsito ou durante a locomoção, enquanto dirigia veículo)
5.1.4 Foi decorrente de outros fatores?
5.2. Sabe a hora da morte?
5.2.1. Hora da morte
5.3 Classificação do Óbito
5.3.1. Descrever outra classificação
5.4. Mecanismo/Meio empregado da morte
5.4.1. Descrever outros mecanismos
5.5.Foi vítima de Tortura?


\subsubsection{Aspectos gerais e éticos do luto}

Um dos principais pontos que se deve ter em conta quando da entrevista com o familiar de vítima de homicídio diz respeito à postura do entrevistador quanto ao luto. Deve-se evitar sempre uma atitude condescendente, ou afirmativa, que vise oferecer soluções ou conforto imediato ao entrevistado. Isso não significa deixar de ser empático, ou de ouvir o que se tem a falar, e sim respeitar os processos do luto de cada um, de forma a entender que não existem explicações ou causas únicas e gerais para a violência vivida. A melhor postura é aquela que admite a incompreensão do entrevistador diante da violência e o coloca como um ouvinte atento.

Por outro lado, a pessoa em luto, se não estiver em um processo patológico, não se furta de auxiliar na constituição da história da morte e do falecido, no sentido de oferecer abertamente informações sobre o ocorrido, principalmente se os sentidos da pesquisa forem dados de forma transparente.

Obviamente, cada caso é diferente, e os ditames éticos gerais devem ser respeitado, e sempre que não houver interesse por parte do entrevistado de falar sobre as circunstâncias da morte ou de dar qualquer outra informação, a vontade desse é soberana.

% Não há o que falar, jamais tentar inferir causas ou motivações.


\subsubsection{Classificação AT}

%inserir a imagem com o quadro explicativo da classificação AT

\subsubsection{Hora da Morte}

\subsubsection{Classificação do Óbito}

\subsubsection{Mecanismo/Meio da Morte}

\subsubsection{Vítima de Tortura?}

\subsection{Dados do Agressor}

6.1. Sabe quem é o Agressor?
6.2. Quantos agressores?
6.1.1. Agressor
6.1.1.1. Era parceiro íntimo?
6.1.1.2. Idade do Agressor
6.1.1.3.Raça/cor do agressor
6.1.1.4. Escolaridade do Agressor
6.1.1.5.Agressor já foi preso?
6.1.1.6. Agressor possui arma de fogo?
6.1.1.7. Agressor tinha relação profissional com a vítima
6.1.1.8. Qual relação
6.1.1.9. Agressor tinha histórico de deficiência ou transtorno mental?
6.1.1.10.Agressor era funcionário do Estado?
6.1.1.11. Cometeu o crime enquanto estava em serviço?
6.1.1.12. Agressor errou de vítima?
6.1.1.13. Agressor agiu mediante promessa ou pagamento?
6.1.1.14. O agressor estava sob efeito de alcool no momento do crime?
6.1.1.15. O agressor estava sob efeito de outras drogas no momento do crime?
6.1.1.16. Qual a conduta do agressor após o crime?
6.1.1.16.1. Descrever conduta outra

\subsection{Violência Doméstica}

7.1. A falecida sofreu violência doméstica na infância?
7.1.1. Quem foi/foram o(s) autor(es) da violência?
7.1.1.2. Descrever outro autor
7.1.2. Tipo de Violência Doméstica sofrida
7.1.2.1. Descrever outro tipo
7.2. A falecida foi testemunha de violência doméstica durante a infância?
7.2.1. Quem foi/foram o(s) autor(es) da violência?
7.2.1.1. Descrever outro autor
7.2.2. Tipo de Violência Doméstica testemunhada
7.2.2.1. Descrever outro tipo

\subsection{Violência por parceiro íntimo}

8.1. A falecida teve um parceiro íntimo nos últimos 30 dias?
8.1.1. Sofreu violência por esse parceiro?
8.1.2. Que tipo de violência?
8.1.2.1. Descrever outro tipo
8.2. A falecida teve um parceiro íntimo no último ano?
8.2.1. Sofreu violência por esse parceiro?
8.2.2. Que tipo de violência?
8.2.2.1. Descrever outro tipo
8.3. A falecida teve um parceiro íntimo ao longo da vida?
8.3.1. A quanto tempo? 
8.3.2. Sofreu violência por esse parceiro?
8.3.3. Que tipo de violência?
8.3.3.1. Descrever outro tipo
8.4. A falecida fez denúncia da violência às autoridades?
8.5. A falecida estava sob medida protetivas?

\subsection{Violência na Infância por parceiro íntimo}

9.1. O parceiro íntimo da falecida sofreu violência durante a infância?
9.1.1. Quem foi o autor da violência?
9.1.1.1. Descrever outro autor
9.1.2. Que tipo de violência?
9.1.2.1. Descrever outro tipo
9.2.O parceiro íntimo da falecida foi testemunha de violência durante a infância?
9.2.1. Quem foi o autor da violência?
9.2.1.1. Descrever outro autor
9.2.2. Que tipo de violência?
9.2.2.1. Descrever outro tipo

\subsection{Gravidez}

10.1 A falecida estava grávida?	10.1 Está grávida?
10.1.1. Em que trimestre de gravidez estava?	10.1.1. Em que trimestre de gravidez está?
10.2. A falecida esteve grávida nos últimos 12 meses?	10.2. Esteve grávida nos últimos 12 meses?
10.2.1. A falecida sofreu violência pelo parceiro íntimo durante a gravidez?	10.2.1. Sofreu violência pelo parceiro íntimo durante a gravidez?
10.2.1.1 Que tipo de violência	10.2.1.1 Que tipo de violência
10.2.1.1.1. Descrever outro tipo	10.2.1.1.1. Descrever outro tipo
10.2.1.2. Durante qual trimestre a violência ocorreu ou foi mais intensa?	10.2.1.2. Durante qual trimestre a violência ocorreu ou foi mais intensa?

\subsection{Intenção Suicída}

11.1 Você já pensou em tirar a própria vida?
11.2. Você já elaborou um plano para tirar a própria vida?
11.3. Você já tentou tirar a própria vida?
11.4. Algum familiar ou amigo já tentou tirar a própria vida?
11.5. Alguma vez já imaginou que tirar a própria vida poderia resolver seus problemas?

\subsection{Filhos}

12.1. O(a) falecido(a) tinha filhos?
12.1.1. Quantos?
12.1.1.1. Descrição dos filhos
12.1.1.2. Sexo Biológico
12.1.1.3 Idade
12.1.1.4. Filho do(a) falecido(a)?
12.1.1.5. Filho(a) do parceiro(a) do(a) falecido(a)?
12.1.1.6.Filho do agressor?
12.1.1.7. Que pessoas ficaram responsáveis pelo cuidado desse filho após a morte da mãe?
12.1.1.7.1. Descrever outras pessoas

\subsection{Uso de Substâncias Psicoativas}

13.1.O(a) falecido(a) fazia uso de substâncias psicoativas (drogas) lícitas ou ilícitas no último ano?
Indique se, nos últimos 30 dias o(a) falecido (a) fez uso de:
13.1.1. Café e estimulantes a base de cafeína
13.1.2. Álcool
13.1.3. Tabaco
13.1.4. Maconha
13.1.5. Cocaína
13.1.6. Crack
13.1.7. Antidepressivos
13.1.8. LSD
13.1.9. MDMA, Ecstasy
13.1.10. Anfetaminas
13.1.11. Ayahuasca
13.1.12. Inalantes
13.1.13. Antipsicóticos


\subsection{Violência, Guerra às Drogas e Mundos do Crime}

14.1 Nos últimos 30 dias o(a) falecido(a) sofreu algum tipo de ameaça?
14.2. Nos últimos 30 dias, o(a) falecido(a) teve receio de ser morto por criminosos ou por policiais?
14.3. O(a) falecido(a) tinha nos últimos 30 dias dívidas pelo uso de drogas?
14.4. Nos últimos 30 dias, o(a) falecido(a) vivenciou violência direta oriunda do confronto entre policiais (em serviço ou não) ou traficantes/criminosos?
14.5. Nos últimos 30 dias, o(a) falecido(a) possuía relação direta com esses grupos ( dívidas, obrigações, pertencimento )?
14.6. Nos últimos 30 dias o(a) falecido(a) usava, comercializava, produzia ou distribuía drogas ilícitas recentemente?
14.7. O(a) falecido(a) já foi preso pelo uso, produção, comercialização ou distribuição de drogas ilícitas?
14.8. O(a) falecido(a) já foi condenado/processado criminalmente?
14.8.1. Qual crime?
14.8.2. Foi preso?
14.9. Estava preso?
14.8.2.1. Tempo de condenação
14.8.2.2. Quanto tempo estava em liberdade?
14.10. O(a) falecido(a) já foi internado pelo uso de álcool e outras drogas?
14.11. Qual(is) local(is)?

\subsection{Opinião sobre políticas de drogas}

As drogas, e por extensão, aquele que as usa, vende, planta ou produz, devem ser percebidas como inimigas ou indesejáveis em si mesmas
Ações militares ou policiais são um recurso principal ou fundamental para se lidar com o problema das drogas ilícitas
As drogas ilegais e seu uso podem e devem ser erradicados
A solução para a dependência ou abuso de drogas é a abstinência, ou seja, cessar completamente o seu uso
Ações para prender pessoas (na cadeia ou em serviços de tratamento) devem ser incentivadas para resolver o problema das drogas
Não existem usos benéficos para as drogas
A prevenção no campo das drogas deve sempre inibir o seu uso, sem relativizar seus riscos
Os indivíduos que usam, vendem, plantam ou produzem drogas devem ser respeitados em sua condição de pessoa humana, ainda que seus atos configurem crimes
Ações militares ou policiais não devem fazer parte dos recursos para se lidar com a questão das drogas
O uso de drogas faz parte da história da humanidade e seu uso é impossível de ser erradicado
Existem soluções para o uso problemático de drogas além de parar completamente o uso (abstinência)
O aprisionamento de pessoas, em qualquer modalidade, deve ser abolido enquanto forma de resolver a questão das drogas.
As drogas, ainda que ilícitas, podem possuir usos benéficos
A prevenção no campo das drogas deve considerar os riscos e danos no contexto de cada sujeito, não devendo necessariamente coibir qualquer tipo de uso

\subsection{Violência no Trânsito}

16.1 No último ano o(a) falecido(a) foi vítima de acidente de trânsito?
16.1.1 Qual a condição do(a) falecido durante o acidente?
16.1.1.1. Descrever outra condição
16.1.2. Qual o local do acidente?
16.1.3. Rota: o(a) falecido(a) partia de onde?
16.1.4. Rota: onde o(a) falecido(a) pretendia chegar?
16.1.5. Quem foi o responsável, na sua opinião, pelo acidente?
16.1.6. O(a) falecido(a) teve alguma sequela ou incapacidade decorrente de acidente de trânsito?

\subsection{Dados Policiais}

17.1. A polícia civil (sem farda) te procurou?
17.2. A políicia fardada (PM ou GM\_ te procurou?

\subsection{Dados complementares}

18.1. Conhece algum caso de pessoa que desapareceu?
18.1.1. Descrever
18.2. Conhece algum “trabalhador ilícito”?
18.2.1. Descrever
18.3. Telefone contato What's App
18.4. Telefone 2

\subsection{Dados Espaciais}

20.1. Local de Procedência do(a) falecido(a)
20.2. Tempo que o(a) falecido(a) residia em Campinas
20.3. Local do evento que produziu a morte
20.4. Tipo de local
20.4.1. Descrever tipo de local
20.5. Endereço de residência
20.6. Endereço de Ficância
20.7.Tipo de local
20.7.1. Descrever tipo de local

\subsection{Deficiência ou Transtorno}

21.1. O(a) falecido(a) possuía algum tipo de deficiência ou transtorno
21.2. Qual tipo de deficiência?
21.2.1. Descrever outro tipo de deficiência

\subsection{Homicídios}

O homicídio possui um caráter duplo na pesquisa: ele pode ser visto tanto como um crime quanto como uma desfecho de saúde. O resultado, em ambos os casos é o mesmo, o fim da vida de uma pessoa pela ação de outra. No entanto, do ponto de vista do direito a punição do agente é aquilo que dá fundamento à existência do crime. Assim, é mais importante saber acerca da intencionalidade do agente na prática, isto é, se o criminoso fez o que fez porque quis. Já no âmbito da saúde, muitas vezes a natureza da lesão ou do contexto da agressão são mais importante.

Dessa forma, muitos casos visto como homicídio no olhar da saúde podem acabar sendo vistos como ações dentro da lei pelo Direito. Isso será objeto de pesquisas no Grupo, então essas sutis diferenças de julgamento e registro das ocorrências não devem ser ignorada.

Por essa razão, durante a entrevista, elementos que conduzam a uma história acerca da intenção do agente criminoso devem ser levadas em consideração. Entretanto, se houverem elementos na história que permitam de fato "descobrir" o agente ou sua intenção de forma a colaborar para possível investigação criminal, isso deve ser discutido com a equipe para tomada de medidas cabíveis, em razão de possível dever ético de denunciar ou intervir.

\subsection{Feminicídios e Violência Contra a Mulher}

% Feminicido, definição de Carcedo
% Tipos de violência Contra a mulher
% Fatores de risco associados

\subsection{Suicídio}

% Formas de se abordar

\subsection{Violência contra os idosos}

\subsection{Violência contra as crianças}

\subsection{Acidente do Trabalho}

\subsection{Guerra às Drogas e Uso de Substâncias Psicoativas}

Primeiramente, o pesquisador deve ter em mente a predominância de usos de drogas não causadores de dependência. Ainda que o senso comum ou mesmo elementos envolvendo a história do caso ou do próprio pesquisador possam apontar para uma avaliação sempre negativa do uso de drogas, deve-se ter em mente que \textbf{o uso problemático de substãncia psicoativos} é menos prevalente do que o \textbf{uso meramente recreativo}. isso significa dizer que devem ser evitadas posturam que reforcem a visão negativa do entrevistado sobre o papel negativo da droga no caso.

Casos de overdose deveriam ser classificados como causas externas no capítulo dedicados à intoxicação, no entanto são normalmente sub-notificados. O pesquisador deve estar atentos para elementos que indiquem a overdose.

%marcar aula sobre drogas com a turma para explicar os principais efeitos das substâncias.

\subsection{Percepção da Violência e Entrevistas}

Muitas vezes as percepções dos entrevistados sobre a violência podem variar em razão de algum evento violento sofrido tanto pessoalmente quanto no território em que vive. Além disso a mídia exerce um papel importante na constituição dessa percepção, de forma a auxiliar na construção de discursos explicativos sobre os fenômenos vivenciados.

Em razão disso é importante estar atento a elementos na fala que façam referência a esse tipo de influência, tanto explícitos, como por exemplo uma fala que se refere ao ato violento por meio de um programa de televisão visto ou notícias de redes sociais ou grupos de \textit{what's app}, ou até mesmo implícito como em casos que a explicação para a violência repousa em conceitos abrangentes e indeterminados, como o "mundo do crime", ou "as drogas". Não é adequado "suprimir" essa informação, mas deve se estar atento para que os rumos da entrevista não reforcem esse viés do entrevistado, de forma a legitimar ou naturalizar a fala do entrevistado quanto às razões da violência.

\subsection{Desaparecidos}

\subsection{Política}



\section{Referências Bibliográficas}