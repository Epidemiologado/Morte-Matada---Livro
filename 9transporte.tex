
\chapter{Mortes no transporte}

\section{Introdução}

Os acidentes de trânsito são considerados eventos não intencionais que resulta em prejuízo, que podem ser evitáveis e que geram consequências físicas e emocionais. Brasil. Ministério da Saúde. Secretaria de Vigilância em Saúde. Departamento de Vigilância de Doenças e Agravos Não Transmissíveis e Promoção da Saúde. Viva : Vigilância de Violências e Acidentes : 2013 e 2014. Saúde BMd; 2017., e assim como as violências, têm merecido destaque no cenário mundial como um importante problema para a Saúde Pública. 
De acordo os dados da Organização Mundial de Saúde (OMS), os acidentes por transporte ocupam a oitava posição entre as principais causas de mortes. São os responsáveis por 1,35 milhões de morte em 2016 e 50 milhões de feridos em todo o mundo, sendo a primeira causa de óbito nos jovens de 15 a 29 anos. Mais de 90\% das mortes ocorrem em países de renda baixa e média. O grupo mais atingido e com maior risco de acidente corresponde aos pedestres (22\%), os ciclistas (4\%) e os motociclistas (23\%), embora a probabilidade de envolvimento seja alta, isso vai depender das vias, política de segurança e da forma mais habitual de mobilidade segundo região do mundo. WHO. Global status report on road safety. Injury Prevention. 2015;15(4):286.

Na vida dos brasileiros o uso dos veículos automotores está cada vez mais presente, são considerados um bem útil as suas necessidades, mas, em contrapartida, provoca graves problemas sociais quando passa a ser um instrumento que afeta a integridade da pessoa levando-lhe a morte Melo MTd, Nébia Maria Almeida de F. ACIDENTES DE TR NSITO: OS IMPACTOS CAUSADOS NO SETOR PÚBLICO DE SAÚDE E TR NSITO EM RORAIMA/BR. Ambiente: Gestão e Desenvolvimento. 2019;12(3):123-43. 
No Brasil, o trânsito é a segunda maior causa de morte, entre as causas externas, ocorrendo com mais frequência entre jovens e adultos. Nos últimos anos, o número de acidentes de trânsito no Brasil apresentou quedas. Segundo os dados divulgados pelo Sistema de Informação de Mortalidade (SIM) a redução aponta a quase 10\%. Em 2015 o número de mortes por acidente caiu de 39 543, para 38 265 em 2016 (4). Essas tendências de quedas se mantiveram constante e estável. Em 2017 foram 36 430 mortes, 8,3\% a mais que em 2018. Em dados preliminares publicados pelo SIM no ano de 2019, 31 307 mortes ocorreram nas estradas brasileiras Ministério da saúde. Sistema de Informação sobre Mortalidade (SIM). Óbitos por causas externas: banco de dados [Internet]. 2020 [cited 09 set 2020]..

A Organização das Nações Unidas (ONU), estabeleceu o período compreendido entre os anos 2011 a 2020 como a Década de Ação pela Segurança no Trânsito, com o objetivo de estabilizar e posteriormente reduzir as cifras previstas de mortalidade no trânsito, aumentando as atividades no plano nacional, regional e mundial WHO. Global status report on road safety. Injury Prevention. 2015;15(4):286, WHO. Decade of Action for Road Safety 2011-2020 seeks to save millions of lives. 2011. 

Em 2010 as comissões regionais das Nações Unidas ultimaram um projeto sobre o melhoramento da segurança no trânsito no mundo e a consignação de objetivos regionais e nacionais de redução das fatalidades no trânsito, além de proporcionar-se assistência aos governos dos países de baixa e mediana renda na elaboração de seus objetivos (WHO. Decade of Action for Road Safety 2011-2020 seeks to save millions of lives. 2011. ).

Por sua parte, a partir de 2010 o Ministério da Saúde brasileiro em parceria com os estados e municípios, desenvolvem uma ação para o controle dos acidentes de trânsito, principalmente dirigidas ao álcool e à velocidade excessiva com o projeto Vida no Trânsito, tendo como início a participação de cinco capitais: Belo Horizonte, Campo Grande, Curitiba, Palmas e Teresina. (Morais Neto OLd, Silva MMAd, Lima CMd, Malta DC, Silva Júnior JBd. Projeto Vida no Trânsito: avaliação das ações em cinco capitais brasileiras, 2011-2012. 2013,Silva MMAd, Morais Neto OLd, Lima CMd, Malta DC, Silva Júnior JBd. Projeto Vida no Trânsito–2010 a 2012: uma contribuição para a Década de Ações para a Segurança no Trânsito 2011-2020 no Brasil. 2013.). Depois de três, anos foi implementado nas demais capitais e outros municípios.  A importância deste programa se concentra na articulação do setor saúde com o trânsito no cumprimento do Código de Transporte Brasileiro.

A partir de 2013, observou-se uma redução das taxas de mortalidade por acidentes no transporte em todo o Brasil. Esse decréscimo foi associado a maior austeridade na fiscalização da segurança no trânsito, sendo  a Lei seca (Lei n. 12.760, de 20 de dezembro de 2012) um exemplo, que tornou mais rigorosa a fiscalização da condução de veículos sob influência de álcool. Outro acontecimento que pode estar associado seria a crise econômica de 2014, com a redução das atividades econômicas, aumento do desemprego e redução do consumo familiar(de Morais Neto OL, de Aquino ÉC. A MORTALIDADE POR ACIDENTES DE TR NSITO NO BRASIL. Psicologia do Trânsito e Transporte: Manual do Especialista. 2020.).
A partir dessas discussões o propósito deste capítulo é apontar quais são as características e os impactos dos acidentes fatais no trânsito em Campinas.


\section{Resultados}

Noventa e três moradores de Campinas foram vítimas de lesões de transporte fatais no ano de 2019. O coeficiente de mortalidade padronizado para esses eventos foi de 7,7 óbitos para cada 100 mil moradores.

19Com relação ao sexo das vítimas, 83 (89,2\%) eram homens e 10 (10,8\%) eram mulheres, o que implicou em um coeficiente padronizado de mortalidade de 12,1 e 3,7 óbitos para cada 100 mil homens e mulheres, respectivamente. Com relação à cor, 54 (58,1\%) eram amarelos ou brancos enquanto 39 (41,9\%) eram ardos ou pretos.

A Tabela \# mostra estatísticas obtidas ao analisarem-se os dados na forma de um estudo caso-controle, sendo o grupo de casos composto pelos mortos em lesões no transporte e o grupo controle constituído pela população viva amostrada.

Tabela \#: estatísticas obtidas no estudo caso-controle tendo como casos os falecidos em acidentes de transporte e como controles a população viva amostrada.

Pode-se interpretar esses resultados da seguinte forma:
\begin{itemize}
    \item Um aumento de 0,01 no valor da IDHM em determinada região da cidade está associado a uma redução de 4\% no risco de falecer em decorrência de uma lesão de transporte nessa região.
    \item Um aumento de um ano de escolaridade está associado a uma redução de 10\% do risco de morrer em decorrência de uma lesão de transporte.
    \item Um aumento de um ano de vida está associado a uma diminuição de 2\% do risco de morte em decorrência de uma lesão de transporte.
    \item O uso problemático de álcool está associado a um aumento de 35,9 vezes do risco de morrer em decorrência de acidente de transporte.
    \item O uso recreativo de álcool está associado a um aumento de 2,7 vezes do risco de morrer em decorrência de lesão de transporte.
    \item Homens têm risco de morrer em decorrência de lesão de transporte 8,7 vezes maior do que mulheres.
    \item Exercer trabalho remunerado no último mês está associado a um aumento de 6,7 vezes do risco de morte por lesão de transporte.
\end{itemize}

A Figura \# mostra a distribuição dos locais de ocorrência das lesões no transporte que causaram a morte desses 93 moradores de Campinas segundo as regiões pobre, média e rica da cidade. Levando-se em consideração as populações dessas regiões, os coeficientes padronizados foram 9,7; 9,1 e 4,9 por cem mil habitantes, respectivamente. Assim, o risco de morrer devido a uma lesão de transporte entre moradores das regiões pobre e média, em relação a moradores da região rica do município foi estimado em 2,0 e 1,8, respectivamente.

A Tabela \# apresenta coeficientes de mortalidade padronizados e estimativas de risco relativo de morte decorrentes de lesões no transporte segundo regiões da cidade e sexo.

A Figura \# apresenta a distribuição espacial do risco de morte decorrente de lesões no transporte no município de Campinas em relação ao risco médio na cidade, ressaltando uma nítida área de proteção no centro do município.

\section{Discussão}

Conforme os dados já supracitados no presente estudo, houve uma maior prevalência de homens envolvidos em acidentes de trânsito do que mulheres. Esse resultado concorda com a maioria dos demais autores estudados, que também encontraram maior número de homens envolvidos em acidentes de trânsito, a maioria com taxas acima de 60\%. Nunes HRdC. Fatores associados ao óbito por acidentes de trânsito no Brasil: Uma série de estudos com dados secundários. 2019. Marín L, Queiroz MS. A atualidade dos acidentes de trânsito na era da velocidade: uma visão geral. Cadernos de Saúde Pública. 2000;16:7-21.

Na literatura cientifica pode-se evidenciar que a maior ocorrência de vítimas do sexo masculino em acidentes se deve a diferença de gênero, manifestada pela exposição histórica do sexo masculino a motorização, como representação da hegemonia da masculinidade.  Özkan T, Lajunen T. What causes the differences in driving between young men and women? The effects of gender roles and sex on young drivers' driving behavior and self-assessment of skills. Transportation Research Part F: Traffic Psychology and Behaviour. 2006;9:269-77. Factor R, Mahalel D, Yair G. Inter-group differences in road-traffic crash involvement. Accident Analysis & Prevention. 2008;40(6):2000-7.

Em relação ao veículo mais envolvido nos acidentes de transporte, predominaram as motocicletas, o que corroborou com os resultados encontrados, apresentando um elevado percentual de participação de motocicletas (58\%) em relação aos carros (21,6\%), pedestres (21,6\%) e van (1\%) envolvidos nos acidentes de trânsito. Em nível nacional, verificou-se que os usuários de motocicletas também se destacaram como vítimas de acidentes de trânsito. Dados do Ministério da Saúde indicam que os ocupantes de motocicletas correspondem à maioria das vítimas nos registros dos sistemas de informação do SUS, desde o atendimento ambulatorial até nas cifras de óbitos. Ministério da saúde. Sistema de Informação sobre Mortalidade (SIM). Óbitos por causas externas: banco de dados [Internet]. 2020 [cited 09 set 2020].

As características do município de Campinas atendem aos determinantes, apontados na literatura, na incidência de acidentes com motociclistas. Segundo o Departamento Nacional de Trânsito (DENATRAN) a frota de motos teve um aumento de 46,1\% nos últimos dez anos. Brasil. Departamento Nacional de Trânsito (Denatran) 2020 [cited 2020 20/10]. Available from: http://www.denatran.gov.br/frota.htm.

Considera-se que o atual momento econômico, assim como as diferentes formas de financiamento, possibilita a compra de uma motocicleta. Associado ao crescente número desse veículo nas ruas e avenidas de Campinas, o destaque é o desrespeito às leis de trânsito por parte dos motociclistas, assim como o excesso de velocidade e o uso de álcool, fatores que determinam a gravidade dos casos de acidentes.

O uso problemático de álcool é uma das principais causas associadas aos acidentes trânsito. Oliveira JBd, Kerr-Corrêa F, Lopes ÍC, Vitti Junior W, Nunes HRdC, Lima MCP. Alcohol use and risk of vehicle accidents: cross-sectional study in the city of São Paulo, Brazil. Sao Paulo Medical Journal. 2020(AHEAD). Além disso, outros fatores apresentaram um destaque significativo, como problemas nas rodovias (buracos, falta de iluminação, sinalização deficiente, etc.) e questões relacionadas ao trajeto (realizar ultrapassagens, dirigir na contramão, conversão errada, excesso de velocidade, não prestar socorro, não usar capacete ou cinto de segurança, etc.).

Em relação a causa e gravidade dos acidentes, as pessoas alcoolizadas tendem a perder o controle do veículo com maior facilidade, principalmente quando se trata de uma motocicleta, e o excesso de velocidade está relacionado à gravidade do acidente. Este dado encontrado é de fundamental importância, visto que, no Brasil, as motocicletas representam o meio de transporte socialmente relevante, especialmente para a classe trabalhadora, para prestação de serviços ou meio de transporte próprio. Miziara ID, Miziara CSMG, Rocha LE. Acidentes de Motocicletas e sua relação com o trabalho: revisão da literatura. Saúde, Ética & Justiça. 2014;19(2):52-9.

Em relação a classe trabalhadora, as mortes relacionadas ao Acidente do Trabalho fatal, 76,7\% foram decorrentes de acidentes relacionados ao transporte, podendo ser colisões, atropelamentos ou quedas de veículos, assim como desequilíbrio ou quedas do trabalho durante sua locomoção a pé (trajeto de casa ao trabalho ou trajeto do trabalho para casa após expediente). Os acidentes resultantes estritamente da execução de tarefa e atividades do trabalho foi de 23,2% e a maioria desses trabalhadores eram motoboys.
No Brasil entre os anos de 2009 e 2016, ocorreu um aumento da incidência dos acidentes de trajeto de 88,7 para 105,8, atingindo predominantemente ao sexo masculino entre as idades de 20 e 29 anos. Cunha AA, Corona RA, Silva DG, Fecury AA, de Mattos Dias CAG, Araújo MHM. Tendência na incidência de acidentes de trajeto em trabalhadores no Brasil entre 2009 e 2016. RevBrasMedTrab. 2019;17(2):490-8

O aumento do número de motoboys sugere uma participação na satisfação das necessidades sociais contemporâneas e das exigências do mercado. Contudo, os resultados dos acidentes fatais envolvendo motoboys, refletem a susceptibilidade desses trabalhadores a esses eventos, diretamente relacionados com a sua profissão, muitos deles com precárias condições de trabalho e alta exposição a circunstâncias de risco. Soares DFPdP, Mathias TAdF, Silva DWd, Andrade SMd. Motociclistas de entrega: algumas características dos acidentes de trânsito na região sul do Brasil. Revista Brasileira de Epidemiologia. 2011;14:435-44. 

Outro fator associado ao envolvimento dos acidentes de trânsito foi a desigualdade social e do espaço urbano. O deslocamento das pessoas e bens na cidade visa um sistema complexo de atividades econômicas e sociais, e neste caso, as pessoas pobres e idosos com limitações estariam nas faixas inferiores de mobilidade em relação à classe de renda mais alta. As variáveis tempo e custo da viagem pesam mais nesta população, agravando as desigualdades. Pero V, Mihessen V. Mobilidade urbana e pobreza no Rio de Janeiro. 2013. Desta maneira a segregação é ainda mais  aos pobres para as periferias e colocando as elites em lugares de melhor acesso. A globalização automobilística, atingem desigualmente as classes sociais, sendo participe com a má oferta de transporte coletivo e o aumento da mortalidade dos mais pobres. RIBEIRO RH, SILVA SC. Os acidentes de trânsito como expressão da globalização perversa do uso do território. 
