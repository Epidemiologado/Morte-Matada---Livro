\chapter{Autópsia Verbal}

Investigar sistematicamente causas de morte por meio de entrevistas com vivos é uma prática antiga. Já na Londres do século XVII, tão logo alguém morria, visitadoras (conhecidas na comunidade como death searchers) eram enviadas para investigar as causas do óbito em entrevistas com familiares, vizinhos e colegas de ofício do morto \citep{garenne2006potential, spence2016accidents}. Modernamente, trabalhos pioneiros desenvolvidos ao longo dos anos 1950 e 1960 na Índia, em Bangladesh e em alguns países do continente africano utilizaram entrevistas sistematizadas para a investigação de causas de morte. O termo autópsia verbal foi pela primeira vez usado nesse contexto no desenvolvimento do Narangwal Research Project \citep{bang1992diagnosis}, que estudou a interação entre desnutrição e doenças infecciosas da infância na região de Punjab (Índia) nos anos 1960 \citep{taylor1983child}. A Organização Mundial da Saúde, em 2007, assim se referiu ao método:

\begin{quote}
A autópsia verbal é uma entrevista realizada com familiares e/ou cuidadores do falecido usando um questionário estruturado para identificar sinais e sintomas e outras informações pertinentes que possam ser usadas posteriormente para designar uma provável causa da morte.\citep{abou2007verbal}    
\end{quote}


Inicialmente, autopsias verbais foram realizadas em países em desenvolvimento com sistemas de estatísticas vitais deficientes, predominantemente para investigar mortes causadas por doenças infecciosas em crianças e adultos. Posteriormente, foram utilizadas também para a investigação de mortes por suicídios, acidentes e demais causas externas \citep{gajalakshmi2007suicide}\citep{kahn2000validation}. Atualmente, cerca de 20 países, incluindo China e Índia, usam rotineiramente a autopsia verbal para a identificação de causas básicas de óbito\citep{baiden2007setting}. Estudos analisando a validade do método foram realizados em diversos países\citep{ronsmans1998comparison, world1999standard, chandramohan2001effect, setel2006core}. No Brasil, a autópsia verbal foi usada pela Secretaria de Vigilância em Saúde, do Ministério da Saúde, no Programa de Redução do Percentual de Óbitos com Causas Mal Definidas, nas regiões norte e nordeste, com bons resultados \citep{campos2010uso}. Em estudos na área da Saúde dos Trabalhadores, a autópsia verbal foi usada no Brasil para o esclarecimento de causas de acidentes do trabalho fatais \citep{oliveira1997acidentes, mendes2003verso, CordeiroTrab}. Durante a elaboração deste livro, em meio a pandemia de COVID-19, a Secretaria Estadual de Saúde de São Paulo, considerando que a Organização Mundial de Saúde desaconselha a realização de autópsias para casos suspeitos e confirmados de COVID-19, estabeleceu que a autópsia verbal poderá ser aplicada como um elemento importante para o refinamento ou determinação da causa de óbito\citep{SSESP2020}.

No presente estudo, a autópsia verbal foi o método central na investigação das mortes violentas ocorridas em Campinas em 2019. Não para posterior cunhagem de causas básicas de óbito segundo rubricas da CID-10. Mas sim, para desvelar e contextualizar circunstâncias em que esses óbitos ocorreram. As autópsias verbais alimentaram tanto as narrativas das mortes apresentadas, quanto a análise epidemiológica realizada.

Ao todo, 577 mortes violentas que atingiram moradores de Campinas em 2019 foram investigadas por meio dessa técnica. A equipe de entrevistadores foi composta por 14 alunos de Pós-Graduação em Saúde Coletiva da Unicamp, das áreas de concentração em epidemiologia e ciências sociais. Ao longo de 2019, diariamente essa equipe se deslocou por toda a cidade - de condomínios residenciais luxuosos a vielas que aviltam a condição humana - identificando locais de moradia e convívio dos falecidos e entrevistando seus contatos mais próximos. Vizinhas, vizinhos, amigas, amigos, avós, avôs, mães, pais, tias, tios,esposas, maridos, irmãs, irmãos, filhas, filhos, sobrinhas, sobrinhos falaram. Em muitas ocasiões, diferentes parentes e amigos das vítimas contaram versões conflitantes sobre a morte que presenciaram ou quase. Surpreendentemente, ouve poucas recusas por parte dos amigos e parentes em participar de dolorosas conversas como essas.

No questionário havia perguntas fechadas sobre variáveis sócio-demográficas. Apontava-se também o exato lugar de ocorrência da lesão que desencadeou o óbito. A essas informações, juntaram-se dados obtidos nas necrópsias realizadas no Instituto Médico Legal de Campinas e outras cidades, bem como em boletins de ocorrência e na mídia impressa e falada. Dados advindos do sistema judiciário agregaram muita informação às histórias de morte.

Entretanto, diferentemente da autópsia verbal desenvolvida e preconizada pela OMS, que adota como instrumento um questionário estruturado, a riqueza da abordagem se revelou no encaminhamento das questões abertas aplicadas, espaço em que os entrevistados discorreram livremente. Como era fulano? O que fazia? Com quem se relacionava? Em que trabalhava? Como se divertia? E dentre tantas perguntas, uma em particular: como e por que fulano morreu? Todos falaram. Falaram muito. Falaram mais do que pudemos captar. Esse livro resulta da interação afetiva e existencial de um sem número de subjetividades: a da equipe de entrevistadores e a dos que conviviam com seus parentes/vizinhos/amigos recém falecidos.

\section{Fontes Complementares de Dados do Sistema Judiciário}

...Bellini vai desenvolver...

\section{Mídia}

...Thamiris e Lígia vão desenvolver...

A imprensa, tanto escrita quanto falada, noticia também a ocorrência de mortes violentas. Em que pese a cautela com o comportamento sensacionalista de muitos veículos de divulgação – que frequentemente supervalorizam aspectos ainda em investigação – alguns autores têm indicado que jornais são uma importante fonte de qualificação e complementação de informações sobre óbitos por causas externas \citep{souza2006acidentes, villela2012utilizaccao}.

\section{IML}

...Paulo vai desenvolver...

\section{Grupo Controle}

Ao nos depararmos com 584 mortes violentas ocorridas em apenas um ano numa cidade como Campinas, ao menos duas perguntas emergem. Além das mortes, o que mais diferencia esse grupo da população onde viviam? Estariam essas características implicadas nas violências ocorridas?

Tais perguntas habitam o campo da epidemiologia, área da saúde que estuda relações entre exposições das mais diversas naturezas e fenômenos do processo saúde/doença ocorrendo em populações humanas. Nesse ambiente, o entendimento de relações causais que poderiam explicar parcialmente a ocorrência dessas mortes é tentado comumente por meio da execução de estudos caso-controle. O desenvolvimento desses estudos constitui a maior contribuição metodológica da epidemiologia\citep{cordeiro2005mito}.

Responder à primeira questão requer conhecimento aprofundado da população onde viviam os mortos aqui identificados. Para tanto, os estudos caso-controle propiciam uma maneira eficiente de obter estimativas da distribuição basal de exposições na população fonte de violências, dispensando o consumo de tempo e recursos financeiros/administrativos que a tradicional abordagem de enumeração e seguimento de um grande conjunto de indivíduos, classificados de acordo com níveis de exposição, exigiria. O grupo controle é a pedra de toque nesta tarefa. É uma representação simplificada e conveniente do universo que gerou as mortes estudadas. Abaixo é descrito como esse grupo foi constituído.

A companhia distribuidora de água potável de Campinas (SANASA) refere que sua rede acessa 99,81\% da população da cidade \citep{SANASA2020}. Sendo assim, ela possui um cadastro praticamente universal de moradias do município. Isso implica que, à parte os moradores de rua, toda a população da cidade pode ser acessada por meio desse cadastro.

Mediante um protocolo de colaboração estabelecido com a SANASA, obteve-se uma amostra aleatória simples de 800 domicílios residenciais de Campinas ligados à rede de distribuição de água. Os mesmos entrevistadores que procederam as autópsias verbais acessaram essas residências. Em um primeiro contato com um morador do domicílio amostrado, os objetivos do estudo foram apresentados. Mediante sorteio, dentre todos os moradores do domicílio sorteado, um foi escolhido. Após consentir em participar do estudo, esse morador foi entrevistado. Todas as questões feitas nas autópsias verbais das vítimas de violência foram reproduzidas para os integrantes do grupo controles, exceto a investigação específica sobre a morte.

Dos 800 endereços residenciais de Campinas amostrados, 29 (3,6\%) foram descartados sem reposição. Isso se deu por diferentes motivos:

\begin{itemize}
    \item a) O morador que atendeu o primeiro contato da equipe se recusou a participar do estudo.
    \item b) O morador sorteado para a entrevista se recusou a participar do estudo.
    \item c) O morador sorteado para a entrevista não se encontrava em casa em três visitas consecutivas pré-agendadas.
    \item d) O domicílio sorteado encontrava-se fechado em três visitas realizadas em horários diferentes.
    \item e) O acesso do entrevistador ao domicílio sorteado foi proibido na portaria do prédio ou condomínio.
\end{itemize}

Assim, o grupo controle foi composto por 771 moradores de Campinas, sorteados nos domicílios amostrados, que consentiram em participar do estudo mediante assinatura de Termo de Consentimento Livre e Esclarecido, regulamentado pelo Comitê de Ética em Pesquisa da Faculdade de Ciências Médicas da UNICAMP. A Figura \# apresenta a distribuição espacial do local de moradia dos integrantes do grupo controle.

%Inserir Figura